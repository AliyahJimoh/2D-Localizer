\documentclass{article}

\usepackage{tabularx}
\usepackage{booktabs}

\title{Problem Statement and Goals\\2-D Localizer}

\author{Aliyah Jimoh}

\date{January 17th, 2024}

\input{../Comments}
%% Common Parts

\newcommand{\progname}{2D Localizer} % PUT YOUR PROGRAM NAME HERE
\newcommand{\authname}{Aliyah Jimoh} % AUTHOR NAMES                  

\usepackage{hyperref}
    \hypersetup{colorlinks=true, linkcolor=blue, citecolor=blue, filecolor=blue,
                urlcolor=blue, unicode=false}
    \urlstyle{same}
                                


\begin{document}

\maketitle

\begin{table}[hp]
\caption{Revision History} \label{TblRevisionHistory}
\begin{tabularx}{\textwidth}{llX}
\toprule
\textbf{Date} & \textbf{Developer(s)} & \textbf{Change}\\
\midrule
17 January 2024 & Aliyah Jimoh & Initial Draft\\

\bottomrule
\end{tabularx}
\end{table}

\section{Problem Statement}
Mobile robots can be used to travel certain terrain or complete difficult tasks that could jeopardize the safety of the operators. Using sensors to let the robot know where it is, make it actively react or change its trajectory can help ensure the safety of the robot itself.
\subsection{Problem}
With environments that are closed space or indoors, it would be hard to get a GPS signal to locate the robot or see if the task has been completed. Testing the robot's trajectory and and algorithm in the actual site without having any way to truly replicate it could risk the robot malfunctioning or having errors making it difficult for retrieving. This project would like to propose a 2-D localization simulator capable of showing the robot's location with along with the robot being able to percept itself.  

\subsection{Inputs and Outputs}

\subsubsection{Inputs}
\begin{itemize}
    \item Room size  
    \item Coordinates of sensors
    \item Predicted Trajectory of robot
\end{itemize}

\subsubsection{Outputs}
\begin{itemize}
    \item Estimated location of robot through sensors with noise (uncertainty)
    \item Estimated location through the robot  
\end{itemize}

\subsection{Stakeholders}
\begin{itemize}
    \item Dr. Matthew Giamou
    \item Roboticists who would want to set up tests for enclosed environments like caves, work cells, etc.
\end{itemize}

\subsection{Environment}
This simulation can be accessed through any operating system which includes Windows 11 and higher, MacOS, and Linux operating systems.

\section{Goals}

\section{Stretch Goals}

\section{Challenge Level and Extras}

\wss{State your expected challenge level (advanced, general or basic).  The
challenge can come through the required domain knowledge, the implementation or
something else.  Usually the greater the novelty of a project the greater its
challenge level.  You should include your rationale for the selected level.
Approval of the level will be part of the discussion with the instructor for
approving the project.  The challenge level, with the approval (or request) of
the instructor, can be modified over the course of the term.}

\newpage{}

\end{document}