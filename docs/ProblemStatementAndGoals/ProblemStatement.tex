\documentclass{article}

\usepackage{tabularx}
\usepackage{booktabs}

\title{Problem Statement and Goals\\2-D Localizer}

\author{Aliyah Jimoh}

\date{January 17th, 2025}

%% Comments

\usepackage{color}

\newif\ifcomments\commentstrue %displays comments
%\newif\ifcomments\commentsfalse %so that comments do not display

\ifcomments
\newcommand{\authornote}[3]{\textcolor{#1}{[#3 ---#2]}}
\newcommand{\todo}[1]{\textcolor{red}{[TODO: #1]}}
\else
\newcommand{\authornote}[3]{}
\newcommand{\todo}[1]{}
\fi

\newcommand{\wss}[1]{\authornote{blue}{SS}{#1}} 
\newcommand{\plt}[1]{\authornote{magenta}{TPLT}{#1}} %For explanation of the template
\newcommand{\an}[1]{\authornote{cyan}{Author}{#1}}

%% Common Parts

\newcommand{\progname}{2D Localizer} % PUT YOUR PROGRAM NAME HERE
\newcommand{\authname}{Aliyah Jimoh} % AUTHOR NAMES                  

\usepackage{hyperref}
    \hypersetup{colorlinks=true, linkcolor=blue, citecolor=blue, filecolor=blue,
                urlcolor=blue, unicode=false}
    \urlstyle{same}
                                


\begin{document}

\maketitle

\begin{table}[hp]
\caption{Revision History} \label{TblRevisionHistory}
\begin{tabularx}{\textwidth}{llX}
\toprule
\textbf{Date} & \textbf{Developer(s)} & \textbf{Change}\\
\midrule
17 January 2025 & Aliyah Jimoh & Initial Draft\\
21 January 2025 & Aliyah Jimoh & Modifying draft according to issues\\

\bottomrule
\end{tabularx}
\end{table}

\section{Problem Statement}
Mobile robots can be used to complete difficult tasks or traverse challenging environments with industrial hazards. Sensors can be placed on the robot along with its surroundings which can enable the robot to know where it is, make proactive decisions or change its trajectory.

\subsection{Problem}
Environments that are enclosed or indoors would not have access to a GPS signal to locate the robot or see if its work has been completed. Testing the robot's trajectory and algorithm in the actual site without having any way to truly track its movements could complicate the retrieval process if something were to malfunction or if an obstacle is unaccounted for that stops the robot from returning. This project proposes a 2-D localization solution that uses sensors to accurately display the robot's location as it autonomously carries out their assigned tasks while also having the robot locate itself.

\subsection{Inputs and Outputs}

\subsubsection{Inputs}
The inputs of this project include a 2-D map of the environment, the coordinates of each range sensor and fiducial marker relative to the environment, and the noisy measurements made from the sensors on the robot.

\subsubsection{Outputs}
The outputs expected are estimates of the robot's poses computed with sensor measurements through the range sensors in the environment and the sensors on the robot.

\subsection{Stakeholders}
An immediate stakeholder for this project is Dr. Matthew Giamou of the Autonomous Robotics and Convex Optimization (ARCO) Lab due to this being part of a graduate research project that aims to optimize sensor placement for mobile robots. The secondary stakeholders would be academic and industrial roboticists who would want to set up tests for enclosed environments like caves, industrial cells, etc.

\subsection{Environment}
This localizer can be accessed through Linux operating systems.

\section{Goals}
\begin{enumerate}
    \item The localizer will support models of noisy range sensors and cameras that can perceive fiducial markers with known and fixed poses in the environment.
    \item The localizer will return an estimate of the robot localization.
    \item The estimated poses will be visually represented on the inputted map.
\end{enumerate} 

\section{Stretch Goals}
\begin{enumerate}           
    \item The localizer will display the robot's poses from the sensors' feedback.
    \item The simulation will display an animation of the robot's trajectory through the environment with the sensors placed.
\end{enumerate}

\section{Challenge Level and Extras}
This project is expected to have an advanced research level which can be seen from the implementation and the topic. Setting up the robot's movement, displaying the trajectory, coordinating the sensors' measurements, and finding a way to animate the output would definitely add difficulty to this project, however, this is not a niche topic meaning that there are papers or libraries available to draw inspiration from.

\newpage{}

\end{document}