\documentclass[12pt]{article}

\usepackage{amsmath, mathtools}
\usepackage{amsfonts}
\usepackage{amssymb}
\usepackage{graphicx}
\usepackage{colortbl}
\usepackage{xr}
\usepackage{hyperref}
\usepackage{longtable}
\usepackage{xfrac}
\usepackage{tabularx}
\usepackage{float}
\usepackage{siunitx}
\usepackage{booktabs}
\usepackage{caption}
\usepackage{pdflscape}
\usepackage{afterpage}


\usepackage[round]{natbib}

%\usepackage{refcheck}

\hypersetup{
    bookmarks=true,         % show bookmarks bar?
      colorlinks=true,       % false: boxed links; true: colored links
    linkcolor=red,          % color of internal links (change box color with linkbordercolor)
    citecolor=green,        % color of links to bibliography
    filecolor=magenta,      % color of file links
    urlcolor=cyan           % color of external links
}

\input{../Comments}
%% Common Parts

\newcommand{\progname}{2D Localizer} % PUT YOUR PROGRAM NAME HERE
\newcommand{\authname}{Aliyah Jimoh} % AUTHOR NAMES                  

\usepackage{hyperref}
    \hypersetup{colorlinks=true, linkcolor=blue, citecolor=blue, filecolor=blue,
                urlcolor=blue, unicode=false}
    \urlstyle{same}
                                


% For easy change of table widths
\newcommand{\colZwidth}{1.0\textwidth}
\newcommand{\colAwidth}{0.15\textwidth}
\newcommand{\colBwidth}{0.82\textwidth}
\newcommand{\colCwidth}{0.1\textwidth}
\newcommand{\colDwidth}{0.05\textwidth}
\newcommand{\colEwidth}{0.8\textwidth}
\newcommand{\colFwidth}{0.17\textwidth}
\newcommand{\colGwidth}{0.5\textwidth}
\newcommand{\colHwidth}{0.28\textwidth}

% Used so that cross-references have a meaningful prefix
\newcounter{defnum} %Definition Number
\newcommand{\dthedefnum}{GD\thedefnum}
\newcommand{\dref}[1]{GD\ref{#1}}
\newcounter{datadefnum} %Datadefinition Number
\newcommand{\ddthedatadefnum}{DD\thedatadefnum}
\newcommand{\ddref}[1]{DD\ref{#1}}
\newcounter{theorynum} %Theory Number
\newcommand{\tthetheorynum}{TM\thetheorynum}
\newcommand{\tref}[1]{TM\ref{#1}}
\newcounter{tablenum} %Table Number
\newcommand{\tbthetablenum}{TB\thetablenum}
\newcommand{\tbref}[1]{TB\ref{#1}}
\newcounter{assumpnum} %Assumption Number
\newcommand{\atheassumpnum}{A\theassumpnum}
\newcommand{\aref}[1]{A\ref{#1}}
\newcounter{goalnum} %Goal Number
\newcommand{\gthegoalnum}{GS\thegoalnum}
\newcommand{\gsref}[1]{GS\ref{#1}}
\newcounter{instnum} %Instance Number
\newcommand{\itheinstnum}{IM\theinstnum}
\newcommand{\iref}[1]{IM\ref{#1}}
\newcounter{reqnum} %Requirement Number
\newcommand{\rthereqnum}{R\thereqnum}
\newcommand{\rref}[1]{R\ref{#1}}
\newcounter{nfrnum} %NFR Number
\newcommand{\rthenfrnum}{NFR\thenfrnum}
\newcommand{\nfrref}[1]{NFR\ref{#1}}
\newcounter{lcnum} %Likely change number
\newcommand{\lthelcnum}{LC\thelcnum}
\newcommand{\lcref}[1]{LC\ref{#1}}
\newcounter{ucnum} %Unlikely change number
\newcommand{\ltheucnum}{LC\theucnum}
\newcommand{\ucref}[1]{UC\ref{#1}}

\usepackage{fullpage}

\newcommand{\deftheory}[9][Not Applicable]
{
\newpage
\noindent \rule{\textwidth}{0.5mm}

\paragraph{RefName: } \textbf{#2} \phantomsection 
\label{#2}

\paragraph{Label:} #3

\noindent \rule{\textwidth}{0.5mm}

\paragraph{Equation:}

#4

\paragraph{Description:}

#5

\paragraph{Notes:}

#6

\paragraph{Source:}

#7

\paragraph{Ref.\ By:}

#8

\paragraph{Preconditions for \hyperref[#2]{#2}:}
\label{#2_precond}

#9

\paragraph{Derivation for \hyperref[#2]{#2}:}
\label{#2_deriv}

#1

\noindent \rule{\textwidth}{0.5mm}

}

\begin{document}

\title{Software Requirements Specification for 2D Localizer} 
\author{Aliyah Jimoh}
\date{\today}
	
\maketitle

~\newpage

\pagenumbering{roman}

\tableofcontents

~\newpage

\section*{Revision History}

\begin{tabularx}{\textwidth}{p{3cm}p{2cm}X}
\toprule {\bf Date} & {\bf Version} & {\bf Notes}\\
\midrule
2025/02/05 & 1.0 & Initial Draft\\
Date 2 & 1.1 & Notes\\
\bottomrule
\end{tabularx}

~\\

~\newpage

\section{Reference Material}

This section records information for easy reference.

\subsection{Table of Units}

Throughout this document SI (Syst\`{e}me International d'Unit\'{e}s) is employed
as the unit system.  In addition to the basic units, several derived units are
used as described below.  For each unit, the symbol is given followed by a
description of the unit and the SI name.
~\newline

\renewcommand{\arraystretch}{1.2}
%\begin{table}[ht]
\begin{longtable*}{l l l} 
    \toprule		
    \textbf{symbol} & \textbf{unit} & \textbf{SI}\\
    \midrule 
    \si{\metre} & length & metre\\\
    \si{\radian} & angle & radians\\
    \bottomrule
  \end{longtable*}
  %	\caption{Provide a caption}
%\end{table}

% \plt{Only include the units that your SRS actually uses.}

% \plt{Derived units, like newtons, pascal, etc, should show their derivation
%     (the units they are derived from) if their constituent units are in the
%     table of units (that is, if the units they are derived from are used in the
%     document).  For instance, the derivation of pascals as
%     $\si{\pascal}=\si{\newton\per\square\meter}$ is shown if newtons and m are
%     both in the table.  The derivations of newtons would not be shown if kg and
%     s are not both in the table.}

% \plt{The symbol for units named after people use capital letters, but the name
%   of the unit itself uses lower case.  For instance, pascals use the symbol Pa,
%   watts use the symbol W, teslas use the symbol T, newtons use the symbol N,
%   etc.  The one exception to this is degree Celsius.  Details on writing metric
%   units can be found on the 
%   \href{https://www.nist.gov/pml/weights-and-measures/writing-metric-units}
%   {NIST} web-page.}

\subsection{Table of Symbols}

The table that follows summarizes the symbols used in this document along with
their units. The symbols are listed in alphabetical order. 

\plt{Remeber to organize them in order}

\renewcommand{\arraystretch}{1.2}
%\noindent \begin{tabularx}{1.0\textwidth}{l l X}
\noindent \begin{longtable*}{l l p{12cm}} \toprule
\textbf{symbol} & \textbf{unit} & \textbf{description}\\
\midrule 
$i$ & \si[per-mode=symbol] {-} & Index for robot's poses
\\
$j$ & \si[per-mode=symbol] {-} & Index for number of beacons
\\ 
$\eta_j$ & \si[per-mode=symbol] {-} & Sensor noise 
\\
$m$ & \si[per-mode=symbol] {-} & Index for number of FMs
\\ 
$n$ & \si[per-mode=symbol] {-} & Total number of beacons used
\\ 
$S$ & \si[per-mode=symbol] {-} & Set of beacon coordinates
\\
$g_j(x_i)$ & \si[per-mode=symbol] {\metre} & Noisy range measurement of the robot's ith taken from jth beacon
\\ 
$F$ & \si[per-mode=symbol] {-} & Coordinates of FMs
\\ 
$\sigma^2_j$ & \si[per-mode=symbol] {\square\metre} & Noise variance for jth beacon
\\
$\tilde{d_j}$ & \si[per-mode=symbol] {\metre} & Measured distance with noise
\\ 
$C(x_i)$ & \si[per-mode=symbol] {\metre} & Cost Function
\\
$I$ & \si[per-mode=symbol] {\metre^-2} & Total Fisher Information Matrix
\\ 
$\theta$ & \si[per-mode=symbol] {\radian} & Robot angle
\\
$T_{robot}$ & \si[per-mode=symbol] {\square\metre} & surface area over 
which heat is transferred in
\\ 
$\tilde{D}$ & \si[per-mode=symbol] {\square\metre} & coil surface area
\\
$d_j$ & \si[per-mode=symbol] {\square\metre} & surface area over 
which heat is transferred in
\\ 
$T_{env}$ & \si[per-mode=symbol] {\square\metre} & coil surface area
\\
\bottomrule
\end{longtable*}

\subsection{Abbreviations and Acronyms}
\plt{Double check if you need Likely Changes}

\renewcommand{\arraystretch}{1.2}
\begin{longtable*}{l  l} 
  \toprule		
  \textbf{Abbreviation/Acronym} & \textbf{Definition}\\
  \midrule 
  2D & Two-Dimensional \\
  2D Localizer & 2D Localization Solution\\
  A & Assumption\\
  CRLB & Cram\'er-Rao Lower Bound\\
  DD & Data Definition\\
  FIM & Fisher Information Matrix \\
  FM & Fiducial Marker \\
  GD & General Definition\\
  GS & Goal Statement\\
  IM & Instance Model\\
  LC & Likely Change\\
  MLE & Maximum Likelihood Estimation\\
  PDF & Probability Density Function\\
  PS & Physical System Description\\
  R & Requirement\\
  SE(2) & Special Euclidean Group in 2D \\
  SRS & Software Requirements Specification\\
  TM & Theoretical Model\\
  UC & Unlikely Change\\
  \bottomrule
\end{longtable*}

\subsection{Mathematical Notation}
Throughout this document, there will be typographic conventions as 
well as mathematical operators that are used to distinguish different variables and operations.
~\newline

\renewcommand{\arraystretch}{1.2}
\begin{longtable*}{l  l  l} 
  \toprule		
  \textbf{Variable} & \textbf{Definition} & \textbf{Description}\\
  \midrule 
  $\textbf{A}$ & Matrix & Bold capital letter  \\
  $\textbf{a}$ & Vector & Bold lowercase letter  \\
  $a / A$ & Scalar & Italicized uppercase/lowercase symbol \\
$\lVert~\rVert$ & Euclidean (2) Norm~~~~& Vertical brackets \\
  \bottomrule
\end{longtable*}

\newpage

\pagenumbering{arabic}

\section{Introduction}

Mobile robots have been used to traverse areas with various hazards, help collect data whether through its sensors or obtaining samples, and overall complete challenging tasks. Due to their autonomy, there is no need to constantly monitor them as they have their own means with interacting with the environment. However, it could raise some concern when there is no reliable method to track their movements, especially if they are placed in a vastly large area or places that may be difficult to access once they are operational. Risks such as having difficulty retrieving them if there is a malfunction keeping them from returning or possible collisions in the area or with other robots are reasons one would want to find a way to locate them as they carry out their tasks. The program being documented, 2D Localizer, proposes to solve this problem by developing a 2D localization solution that can implement various sensors to accurately localize the robots as they traverse the map provided.

The following section provides an overview of the Software Requirement Specification (SRS) for 2D Localizer. This section explains the purpose of this document, the scope of the requirements, the characteristics of the intended reader, and the organization of the document.

\subsection{Purpose of Document}

The main purpose of this document is to describe the requirements needed to run the localizer. Information such as the constraints, assumptions and theoretical models used will be provided to help readers get a better understanding of the purpose and computations of 2D Localizer. Therefore, it can be used as a reference guide on how to plan and set up the requirements needed by the user to get the desired and accurate results.

\subsection{Scope of Requirements} 

The scope of the requirements includes the robot analyzed being in a controlled environment to help with potential lighting problems for vision sensors used. The sensors used on the robot and environment are specified for modelling purposes. Some models can include a sum through sensors having independent measurements. The noise from the sensors will be considered zero-mean Gaussian for simplicity in calculations.

\subsection{Characteristics of Intended Reader}\label{sec_IntendedReader}

Reviewers of this documentation should have taken a graduate course on linear algebra, estimation theory or matrix computation. They should also have some knowledge in undergraduate statistics and probability.

\subsection{Organization of Document}

The organization of this document follows the template for an SRS for scientific computing software proposed by~\cite{SmithAndLai2005},~\cite{SmithEtAl2007}, and~\cite{SmithAndKoothoor2016}. The presentation follows the standard pattern of presenting goals, theories, definitions, and assumptions. For readers that would like a more bottom up approach, they can start reading the data definitions and trace back to find any additional information they require.

The goal statements are refined to the theoretical models and the theoretical models to the instance models. The data definitions are used to support the definitions of the different models.

%In case I change my mind
% The next section, General System Description, explains the general information that is needed to understand 2D Localizer before going deeper into the document. The Specific System Description provides a refined view of the problem such as clearly defining models and assumptions. The Requirements section 

\section{General System Description}

This section provides general information about the system.  It identifies the
interfaces between the system and its environment, describes the user
characteristics and lists the system constraints.

\subsection{System Context}

The system context is displayed in Figure~\ref{Fig_SystemContext} below. The circles represent the external aspects related to the software which are the users. The rectangle represents the software system being used (2D Localizer) and the arrows explain what information is being passed between the user and the software.

\begin{figure}[h!]
\begin{center}
 \includegraphics[width=0.6\textwidth]{SystemContextFigure.png}
\caption{System Context}
\label{Fig_SystemContext} 
\end{center}
\end{figure}
\begin{itemize}
\item User Responsibilities:
\begin{itemize}
\item Provide inputs including the coordinates of each environmental sensor and the measurements taken from them.
\item Evaluate the inputs to ensure all are their respective types
\end{itemize}
\item 2D Localizer Responsibilities:
\begin{itemize} 
\item Detect data type mismatch, such as a string of characters instead of a
  floating point number
\item Estimate the position and orientation of the robot
\end{itemize}
\end{itemize}

\subsection{User Characteristics}\label{SecUserCharacteristics}

The end user of 2D Localizer should have some familiarity with types of range sensors along with how to read and collect their data. They should also have basic experience with programming.

\subsection{System Constraints}

There are no system constraints.

\section{Specific System Description}

This section first presents the problem description, which gives a high-level
view of the problem to be solved.  This is followed by the solution characteristics
specification, which presents the assumptions, theories, definitions and finally
the instance models.

\subsection{Problem Description}\label{Sec_pd}

2D Localizer is intended to help keep track of mobile robots while carrying out their tasks.

\subsubsection{Terminology and  Definitions}

This subsection provides a list of terms that are used in the subsequent
sections and their meaning, with the purpose of reducing ambiguity and making it
easier to correctly understand the requirements:

\begin{itemize}

\item \textbf{Pose:} Position and orientation of the robot.
\item \textbf{Localization}: Determining where an object is with respect to its environment.
\item \textbf{Fiducial Markers}: Markers placed around the environment for the robot to determine its location.
\item \textbf{Beacon}: A range sensor

\end{itemize}

\subsubsection{Physical System Description}\label{sec_phySystDescrip}

The physical system of the localizer includes the following elements:

\begin{itemize}

\item[PS1:] The mobile robot

\item[PS2:] The beacons placed in the environment

\item[PS3:] The camera sensors on the mobile robot

\item[PS4:] The fiducial markers in the environment

\end{itemize}


% \begin{figure}[h!]
% \begin{center}
% %\rotatebox{-90}
% {
%  \includegraphics[width=0.5\textwidth]{<FigureName>}
% }
% \caption{\label{<Label>} <Caption>}
% \end{center}
% \end{figure}

\subsubsection{Goal Statements}

\noindent Given the imported 2D map, coordinates of all sensors and fiducial markers in the environment, and the noisy measurements of the sensors, the goal statements are:

\begin{itemize}

\item[GS\refstepcounter{goalnum}1\label{GS1}:] Calculate the estimated pose and error of the robot throughout its trajectory from both sensors (environment and robot).
\item[GS\refstepcounter{goalnum}2\label{GS2}:] Display a visual representation of the robot traversing the 2-D map with the sensors placed.

\end{itemize}

\subsection{Solution Characteristics Specification}

The instance models that govern 2D Localizer are presented in
Subsection~\ref{sec_instance}.  The information to understand the meaning of the
instance models and their derivation is also presented, so that the instance
models can be verified.

\subsubsection{Assumptions}\label{sec_assumpt}


This section simplifies the original problem and helps in developing the
theoretical model by filling in the missing information for the physical system.
The numbers given in the square brackets refer to the theoretical model [TM],
general definition [GD], data definition [DD], instance model [IM], or likely
change [LC], in which the respective assumption is used.

\begin{itemize}

\item[A\refstepcounter{assumpnum}1 \label{A_sensors}:]Robot uses camera sensors while the environment uses beacons and fiducial markers (FMs)
\item[A\refstepcounter{assumpnum}2 \label{A_controlled}:]Localizer is used in a controlled environment (i.e., indoors)
\item[A\refstepcounter{assumpnum}1 \label{A_indep}:]Each sensor has independent measurements
\item[A\refstepcounter{assumpnum}1 \label{A_noise}:]Sensor noise is zero-mean Gaussian


\end{itemize}

\subsubsection{Theoretical Models}\label{sec_theoretical}

This section focuses on the general equations and laws that 2D Localizer is based
on.

~\newline
%TM 1
\noindent
\begin{minipage}{\textwidth}
\renewcommand*{\arraystretch}{1.5}
\begin{tabular}{| p{\colAwidth} | p{\colBwidth}|}
\hline
\rowcolor[gray]{0.9}
Number& TM\refstepcounter{theorynum}\thetheorynum\label{T_NRM}\\
\hline
Label &\bf Noisy Range Measurement \\
\hline
Equation& \begin{displaymath}
  g_j(x) = \lVert \mathbf{x_i} - \mathbf{a_j}\rVert + \eta_j
\end{displaymath} \\
\hline
Description &
The equation above gives the noisy range measurement $g_j$ (\si\metre) of the beacons placed in the environment (\aref{A_sensors}) where \\
& $\mathbf{x_i}$ is the position of the robot (\si\metre), \\
& $\mathbf{a_j}$ is the position of the jth beacon placed (\si\metre), and \\
& $\eta_j$ is the noise from the jth beacon (\si\metre) (\aref{A_noise}).
\\
\hline
Source & \citet{Sequeira2024}\\
\hline
Ref.\ By & \ddref{DD_distance}\\
\hline
Preconditions & \aref{A_noise}\\
\hline
Derivation & None\\
\hline
\end{tabular}
\end{minipage}\\

% TM2
~\newline

\noindent
\begin{minipage}{\textwidth}
\renewcommand*{\arraystretch}{1.5}
\begin{tabular}{| p{\colAwidth} | p{\colBwidth}|}
\hline
\rowcolor[gray]{0.9}
Number& TM\refstepcounter{theorynum}\thetheorynum\label{T_FIM}\\
\hline
Label &\bf Fisher Information Matrix \\
\hline
Equation& \begin{displaymath}
  \mathbf{I} \cong \sum_{j=1}^{N}\frac{1}{\sigma_j^2} \frac{\left(\hat{x}-a_j\right) \left( \hat{x}-a_j\right)^T}{\lVert \hat{x}-a_j \rVert^2}
\end{displaymath}\\
\hline
Description &
This equation gives the Fisher Information Matrix which shows how much information the data provides about the robot's position \\
& $\sigma_j^2$ is the sensor variance used in the noise factor\\
& $\hat{x}$ is the estimated position in \iref{IM_MLE}\\
\hline
Source & Citation here \\
\hline
Ref.\ By & \tref{T_CRLB}\\
\hline
Preconditions & \aref{A_noise}, \iref{IM_MLE}\\
\hline
Derivation & \\
\hline
\end{tabular}
\end{minipage}\\

%TM4
~\newline

\noindent
\begin{minipage}{\textwidth}
\renewcommand*{\arraystretch}{1.5}
\begin{tabular}{| p{\colAwidth} | p{\colBwidth}|}
\hline
\rowcolor[gray]{0.9}
Number& TM\refstepcounter{theorynum}\thetheorynum\label{T_CRLB}\\
\hline
Label &\bf Cram\'{e}r-Rao Lower Bound (CRLB)\\
\hline
Equation& \begin{displaymath}
  Var( \hat{x}) \geq \frac{1}{I(\hat{x})}
\end{displaymath}\\
\hline
Description &
This equation gives the CRLB which shows how accurate an estimate of a parameter is given the noise in the measurements by providing a lower bound to the estimate's variance. We are able to use \tref{T_FIM} to get this lower bound \\
& $Var(\hat{x})$ is the variance of the unbiased estimator in \iref{IM_MLE}
\\
\hline
Source & \cite{Barfoot2017} \\
\hline
Ref.\ By & None\\
\hline
Preconditions & \tref{T_FIM}, \iref{IM_MLE}\\
\hline
Derivation & None\\
\hline
\end{tabular}
\end{minipage}\\

%TM5
~\newline

\noindent
\begin{minipage}{\textwidth}
\renewcommand*{\arraystretch}{1.5}
\begin{tabular}{| p{\colAwidth} | p{\colBwidth}|}
\hline
\rowcolor[gray]{0.9}
Number& TM\refstepcounter{theorynum}\thetheorynum\label{T_SE}\\
\hline
Label &\bf Special Euclidean (SE) 2 Transformation \\
\hline
Equation& \begin{displaymath}
  T_{robot} =  
    \begin{bmatrix}
      \mathbf{R}(\theta) & \mathbf{t} \\
      0 & 1\\
    \end{bmatrix}
    =
    \begin{bmatrix}
      \cos\theta & -\sin\theta & x\\
      \sin\theta & \cos\theta & y\\
      0 & 0 & 1\\
    \end{bmatrix}
\end{displaymath}\\
\hline
Description &
This equation shows SE(2) which represents all possible 2D transformations of the robot. Since the robot has cameras placed on it (\aref{A_sensors}), SE(2) can be used to get the position and orientation through the FMs. \\
& $\mathbf{R}(\theta)$ is the rotation represented as a 2$\times$2 matrix\\
& $\mathbf{t}$ is the translation represented as a 2$\times$1 matrix\\
& $[0~1]$ makes sure that the matrix is homogenous 
\\
\hline
Source & \cite{Barfoot2017} \\
\hline
Ref.\ By & \ddref{FluxCoil}, \ddref{FluxPCM}\\
\hline
Preconditions & \\
\hline
Derivation & \\
\hline
\end{tabular}
\end{minipage}\\

~\newline

\subsubsection{General Definitions}\label{sec_gendef}
This section collects the laws and equations that will be used in building the
instance models.

~\newline

\noindent
\begin{minipage}{\textwidth}
\renewcommand*{\arraystretch}{1.5}
\begin{tabular}{| p{\colAwidth} | p{\colBwidth}|}
\hline
\rowcolor[gray]{0.9}
Number& GD\refstepcounter{defnum}\thedefnum\label{GD_PDF}\\
\hline
Label &\bf Probability Density Function (PDF)\\
\hline
% Units&$MLt^{-3}T^0$\\
% \hline
SI Units&\si{\per\metre}\\
\hline
Equation&\begin{displaymath}
  p_j \left( \tilde{d_j}\vert \mathbf{x} \right) = \mathcal{N}\left( \lVert \mathbf{x} - \mathbf{a_j} \rVert, \sigma_j^2 \right)
\end{displaymath}  \\
\hline
Description & This definition is about the Gaussian PDF (\aref{A_noise}) being used in \iref{IM_like}. This provides the PDF of the actual noisy range measurement $\tilde{d_j}$ of beacon $j$ (\ddref{DD_distance}) that is conditioned on the robot's position $\mathbf{x}$.\\
& $\mathcal{N}$ is the normal distribution with $\lVert \mathbf{x} - \mathbf{a_j} \rVert$ as the mean (predicted range) and $\sigma_j^2$ as the variance (sensor noise/uncertainty).
\\
\hline
  Source &\cite{Sequeira2024} \\
  \hline
  Ref.\ By & \iref{IM_like}\\
  \hline
\end{tabular}
\end{minipage}\\

% \subsubsection*{Detailed derivation of simplified rate of change of temperature}

% \plt{This may be necessary when the necessary information does not fit in the
%   description field.}
% \plt{Derivations are important for justifying a given GD.  You want it to be
%   clear where the equation came from.}

\subsubsection{Data Definitions}\label{sec_datadef}

This section collects and defines all the data needed to build the instance
models. The dimension of each quantity is also given.
~\newline

\noindent
\begin{minipage}{\textwidth}
\renewcommand*{\arraystretch}{1.5}
\begin{tabular}{| p{\colAwidth} | p{\colBwidth}|}
\hline
\rowcolor[gray]{0.9}
Number& DD\refstepcounter{datadefnum}\thedatadefnum\label{DD_distance}\\
\hline
Label& \bf Actual Range Measurement\\
\hline
Symbol &$\tilde{d_j}$\\
% Units& $Mt^{-3}$\\
\hline
  SI Units & \si{\metre}\\
  \hline
  Equation& \begin{displaymath}
    \tilde{d_j} = \lVert \mathbf{x_i} - \mathbf{a_j}\rVert + \eta_j
  \end{displaymath}\\
  \hline
  Description & 
  $\tilde{d_j}$ is the actual beacon measurement that will be used compared to \tref{T_NRM} which is just a model. Although it would return a scalar from the sensor, the formulation the equation above. \\
  & $\mathbf{x_i}$ is the position of the robot (\si\metre), \\
  & $\mathbf{a_j}$ is the position of the jth beacon placed (\si\metre), and \\
  & $\eta_j$ is the noise from beacon j(\si\metre) (\aref{A_noise}).
  \\
  \hline
  Sources&\cite{Sequeira2024} \\
  \hline
  Ref.\ By &\dref{GD_PDF}\\
  \hline
\end{tabular}
\end{minipage}\\

\subsubsection{Data Types}\label{sec_datatypes}

\plt{This section is optional.  In many scientific computing programs it isn't
  necessary, since the inputs and outpus are straightforward types, like reals,
  integers, and sequences of reals and integers.  However, for some problems it
  is very helpful to capture the type information.}

\plt{The data types are not derived; they are simply stated and used by other
  models.}

\plt{All data types must be used by at least one of the models.}

\plt{For the mathematical notation for expressing types, the recommendation is
  to use the notation of~\citet{HoffmanAndStrooper1995}.}

This section collects and defines all the data types needed to document the
models. \plt{Modify the examples below for your problem, and add additional
  definitions as appropriate.}

~\newline

\noindent
\begin{minipage}{\textwidth}
\renewcommand*{\arraystretch}{1.5}
\begin{tabular}{| p{\colAwidth} | p{\colBwidth}|}
  \hline
  \rowcolor[gray]{0.9}
  Type Name & Name for Type\\
  \hline
  Type Def & mathematical definition of the type\\
  \hline
  Description & description here
  \\
  \hline
  Sources & Citation here, if the type is borrowed from another source\\
  \hline
\end{tabular}
\end{minipage}\\

\subsubsection{Instance Models}\label{sec_instance}    
This section transforms the problem defined in Section~\ref{Sec_pd} into 
one which is expressed in mathematical terms. It uses concrete symbols defined 
in Section~\ref{sec_datadef} to replace the abstract symbols in the models 
identified in Sections~\ref{sec_theoretical} and~\ref{sec_gendef}.

The goal \gsref{GS1} is solved by \iref{IM_like} and \iref{IM_MLE}.

~\newline

%Instance Model 1

\noindent
\begin{minipage}{\textwidth}
\renewcommand*{\arraystretch}{1.5}
\begin{tabular}{| p{\colAwidth} | p{\colBwidth}|}
  \hline
  \rowcolor[gray]{0.9}
  Number& IM\refstepcounter{instnum}\theinstnum\label{IM_like}\\
  \hline
  Label& \bf Joint Likelihood Function \\
  \hline
  Input&$\mathbf{\tilde{d}}$, $N$\\
  \hline
  Output& $p ( \mathbf{\tilde{d}} \vert \mathbf{x} ) $\\
  \hline
  Equation&\begin{displaymath}
    p \left( \mathbf{\tilde{d}} \vert \mathbf{x} \right) = \prod_{j=1}^{N} p\left( \tilde{d_j}\vert \mathbf{x} \right)
  \end{displaymath}\\
  \hline
  Description& This equation uses the measurements gotten from each beacon (\aref{A_indep}) to calculate the product of how likely $\mathbf{x}$ is the best position to reflect the inputs where \\
  & $\mathbf{\tilde{d}} = [ \tilde{d_1}, \tilde{d_2}, \dots, \tilde{d_N}]$, \\
  & N is the number of beacons, and \\
  & $p( \tilde{d_j}\vert \mathbf{x})$ is the Gaussian PDF from \dref{GD_PDF}
  \\
  \hline
  Sources& \cite{Sequeira2024} \\
  \hline
  Ref.\ By & \iref{IM_MLE}\\
  \hline
\end{tabular}
\end{minipage}\\

~\newline

%IM 2
\noindent
\begin{minipage}{\textwidth}
\renewcommand*{\arraystretch}{1.5}
\begin{tabular}{| p{\colAwidth} | p{\colBwidth}|}
  \hline
  \rowcolor[gray]{0.9}
  Number& IM\refstepcounter{instnum}\theinstnum\label{IM_MLE}\\
  \hline
  Label& \bf Maximum Likelihood Estimation (MLE) \\
  \hline
  Input&$p( \mathbf{\tilde{d}} \vert \mathbf{x} )$\\
  \hline
  Output& $\mathbf{\hat{x}} $\\
  \hline
  Equation&\begin{displaymath}
    \mathbf{\hat{x}} = \underset{x}{\arg\max}\,p( \mathbf{\tilde{d}} \vert \mathbf{x} )
  \end{displaymath}\\
  \hline
  Description&Continuing from \iref{IM_like}, the MLE equation gives an estimate based on the maximum probability calculated from the joint likelihood function.\\
  & $ p( \mathbf{\tilde{d}} \vert \mathbf{x} )$ is the joint likelihood function from \iref{IM_like}\\
  \hline
  Sources& \cite{Sequeira2024} \\
  \hline
  Ref.\ By & \tref{T_FIM}, \tref{T_CRLB}\\
  \hline
\end{tabular}
\end{minipage}\\

~\newline


\subsubsection*{Derivation of ...}

\plt{The derivation shows how the IM is derived from the TMs/GDs.  In cases
  where the derivation cannot be described under the Description field, it will
  be necessary to include this subsection.}

\subsubsection{Input Data Constraints} \label{sec_DataConstraints}    

Table~\ref{TblInputVar} shows the data constraints on the input output
variables.  The column for physical constraints gives the physical limitations
on the range of values that can be taken by the variable.  The column for
software constraints restricts the range of inputs to reasonable values.  The
software constraints will be helpful in the design stage for picking suitable
algorithms.  The constraints are conservative, to give the user of the model the
flexibility to experiment with unusual situations.  The column of typical values
is intended to provide a feel for a common scenario.  The uncertainty column
provides an estimate of the confidence with which the physical quantities can be
measured.  This information would be part of the input if one were performing an
uncertainty quantification exercise.

The specification parameters in Table~\ref{TblInputVar} are listed in
Table~\ref{TblSpecParams}.

\begin{table}[!h]
  \caption{Input Variables} \label{TblInputVar}
  \renewcommand{\arraystretch}{1.2}
\noindent \begin{longtable*}{l l l l c} 
  \toprule
  \textbf{Var} & \textbf{Physical Constraints} & \textbf{Software Constraints} &
                             \textbf{Typical Value} & \textbf{Uncertainty}\\
  \midrule 
  $L$ & $L > 0$ & $L_{\text{min}} \leq L \leq L_{\text{max}}$ & 1.5 \si[per-mode=symbol] {\metre} & 10\%
  \\
  \bottomrule
\end{longtable*}
\end{table}

\noindent 
\begin{description}
\item[(*)] \plt{you might need to add some notes or clarifications}
\end{description}

\begin{table}[!h]
\caption{Specification Parameter Values} \label{TblSpecParams}
\renewcommand{\arraystretch}{1.2}
\noindent \begin{longtable*}{l l} 
  \toprule
  \textbf{Var} & \textbf{Value} \\
  \midrule 
  $L_\text{min}$ & 0.1 \si{\metre}\\
  \bottomrule
\end{longtable*}
\end{table}

\subsubsection{Properties of a Correct Solution}\label{sec_CorrectSolution}

\noindent
A correct solution must exhibit \plt{fill in the details}.  \plt{These
  properties are in addition to the stated requirements.  There is no need to
  repeat the requirements here.  These additional properties may not exist for
  every problem.  Examples include conservation laws (like conservation of
  energy or mass) and known constraints on outputs, which are usually summarized
  in tabular form.  A sample table is shown in Table~\ref{TblOutputVar}}

\begin{table}[!h]
\caption{Output Variables}\label{TblOutputVar}
\renewcommand{\arraystretch}{1.2}
\noindent \begin{longtable*}{l l} 
  \toprule
  \textbf{Var} & \textbf{Physical Constraints} \\
  \midrule 
  $T_W$ & $T_\text{init} \leq T_W \leq T_C$ (by~\aref{A_charge})
  \\
  \bottomrule
\end{longtable*}
\end{table}

\plt{This section is not for test cases or techniques for verification and
  validation.  Those topics will be addressed in the Verification and Validation
  plan.}

\section{Requirements}

This section provides the functional requirements, the business tasks that the
software is expected to complete, and the nonfunctional requirements, the
qualities that the software is expected to exhibit.

\pagebreak[4]

\subsection{Functional Requirements}

\noindent\begin{itemize}

\item[R\refstepcounter{reqnum}\thereqnum\label{R_Inputs}:] Provide the inputs for 2D Localizer which include the 2D map, the coordinates of all sensors and markers located in the environment, and the measurements taken from the sensors.

\item[R\refstepcounter{reqnum}\thereqnum\label{R_OutputInputs}:] 2D Localizer will acquire $\hat{x}$ from the inputs in \rref{R_Inputs}.

\item[R\refstepcounter{reqnum}\thereqnum\label{R_Calculate}:] 2D Localizer will calculate the estimated posed from the measurements provided for \iref{IM_MLE}.

\item[R\refstepcounter{reqnum}\thereqnum\label{R_VerifyOutput}:] 2D Localizer will verify that inputs provided are within their constraints.

\item[R\refstepcounter{reqnum}\thereqnum\label{R_Output}:] 2D Localizer will show a visual animated graph that tracks the mobile robots work while also displaying the sensors' coordinates.

\end{itemize}

\subsection{Nonfunctional Requirements}

This problem priorities the accuracy of the estimations along with its performance when it comes to keeping up with the robot's movements. That being said, the nonfunctional requirements are:

\noindent \begin{itemize}

\item[NFR\refstepcounter{nfrnum}\thenfrnum \label{NFR_Accuracy}:]
  \textbf{Accuracy}: The accuracy of the computed estimations should meet the threshold when comparing to the predicted measurements. 

\item[NFR\refstepcounter{nfrnum}\thenfrnum \label{NFR_Understandibility}:] \textbf{Understandability}: The software should be simple to understand and implement.

\item[NFR\refstepcounter{nfrnum}\thenfrnum \label{NFR_Maintainability}:]
  \textbf{Maintainability}: The software should be simple modify some components when needed.

\item[NFR\refstepcounter{nfrnum}\thenfrnum \label{NFR_Reusability}:]
  \textbf{Usability}: The software should be easy to run and have minimal troubleshooting.
\end{itemize}

\subsection{Rationale}

This program has rationale for the assumptions mentioned in~\ref{sec_assumpt}:
\begin{itemize}
  \item \aref{A_sensors}: Assuming the type of sensors assisted in setting up the type of output variables that is needed for each set of estimates.
  \item \aref{A_controlled}: Referring to \aref{A_sensors}, vision sensors will be used on the robot and lighting could affect the way they detect FMs. Another main reason for this assumption is so that the coordinates of each beacon and FM can be structured around the area the user provides.
  \item \aref{A_indep}: To find the sum and products of variables that need data from each beacon and FM places (\tref{T_FIM}, \iref{IM_like}), having each sensor collect independent data from different positions would help with the overall accuracy the program want to achieve.
  \item \aref{A_noise}: Having a Gaussian noise assumption simplifies computations in the program especially when some sensors used in robotics follow a Gaussian distribution.
\end{itemize}

% \section{Likely Changes}    

% \noindent \begin{itemize}

% \item[LC\refstepcounter{lcnum}\thelcnum\label{LC_meaningfulLabel1}:] \plt{Give
%     the likely changes, with a reference to the related assumption (aref), as appropriate.}

% \end{itemize}

\section{Unlikely Changes}    

\noindent \begin{itemize}

\item[UC\refstepcounter{ucnum}\theucnum\label{UC_noise}:] \aref{A_noise} There are various models that are derived based on the Gaussian noise meaning that it would be detrimental to change.

\end{itemize}

\section{Traceability Matrices and Graphs}

The purpose of the traceability matrices is to provide easy references on what
has to be additionally modified if a certain component is changed.  Every time a
component is changed, the items in the column of that component that are marked
with an ``X'' may have to be modified as well.  Table~\ref{Table:trace} shows the
dependencies of theoretical models, general definitions, data definitions, and
instance models with each other. Table~\ref{Table:R_trace} shows the
dependencies of instance models, requirements, and data constraints on each
other. Table~\ref{Table:A_trace} shows the dependencies of theoretical models,
general definitions, data definitions, instance models, and likely changes on
the assumptions.

\afterpage{
% \begin{landscape}
\begin{table}[h!]
\centering
\begin{tabular}{|c|c|c|c|c|}
\hline
	& \aref{A_sensors}& \aref{A_controlled}& \aref{A_indep}& \aref{A_noise} \\
\hline
\tref{T_NRM}         &X & & &X  \\ \hline
\tref{T_FIM}         & & &X &X  \\ \hline
\tref{T_CRLB}        & & & &  \\ \hline
\tref{T_SE}          &X &X & &  \\ \hline
\dref{GD_PDF}        & & & & X \\ \hline
\ddref{DD_distance}  & & & &X \\ \hline
\iref{IM_like}       & & &X & \\ \hline
\iref{IM_MLE}        & & & & \\ \hline
\ucref{UC_noise}     & & & &X\\
\hline
\end{tabular}
\caption{Traceability Matrix Showing the Connections Between Assumptions and Other Items}
\label{Table:A_trace}
\end{table}
% \end{landscape}
}

\begin{table}[h!]
\centering
\begin{tabular}{|c|c|c|c|c|c|c|c|c|c|c|}
\hline        
	& \tref{T_NRM}& \tref{T_FIM}& \tref{T_CRLB}& \tref{T_SE} & \dref{GD_PDF}& \ddref{DD_distance} & \iref{IM_like} & \iref{IM_MLE}\\
\hline
\tref{T_NRM}     &  & & & & & & & \\ \hline
\tref{T_FIM}     &  & & & & & & &X \\ \hline
\tref{T_CRLB}     &  &X & & & & & & X\\ \hline
\tref{T_SE}     &  & & & & & & &  \\ \hline
\dref{GD_PDF}        &  & & & & & & & \\ \hline
\ddref{DD_distance} & X & & & & & & & \\ \hline
\iref{IM_like}      &  & & & &X &X & & \\ \hline
\iref{IM_MLE}      &  & & & & & &X &\\
\hline
\end{tabular}
\caption{Traceability Matrix Showing the Connections Between Items of Different Sections}
\label{Table:trace}
\end{table}

\begin{table}[h!]
\centering
\begin{tabular}{|c|c|c|c|c|c|c|c|}
\hline
	& \iref{IM_like}& \iref{IM_MLE}& \rref{R_Inputs}& \rref{R_OutputInputs}& \rref{R_Calculate}& \rref{R_VerifyOutput}& \rref{R_Output} \\
\hline
\iref{IM_like}          & & & & & & & \\ \hline
\iref{IM_MLE}           & X& & & & & & \\ \hline
\rref{R_Inputs}         & & & & & & & \\ \hline
\rref{R_OutputInputs}   & & & X& & & & \\ \hline
\rref{R_Calculate}      & & X& & & & & \\ \hline
\rref{R_VerifyOutput}   & & & & & & & \\ \hline
\rref{R_Output}         & & & & & & & \\
\hline
\end{tabular}
\caption{Traceability Matrix Showing the Connections Between Requirements and Instance Models}
\label{Table:R_trace}
\end{table}

% \begin{figure}[h!]
% 	\begin{center}
% 		%\rotatebox{-90}
% 		{
% 			\includegraphics[width=\textwidth]{ATrace.png}
% 		}
% 		\caption{\label{Fig_ATrace} Traceability Matrix Showing the Connections Between Items of Different Sections}
% 	\end{center}
% \end{figure}


% \begin{figure}[h!]
% 	\begin{center}
% 		%\rotatebox{-90}
% 		{
% 			\includegraphics[width=0.7\textwidth]{RTrace.png}
% 		}
% 		\caption{\label{Fig_RTrace} Traceability Matrix Showing the Connections Between Requirements, Instance Models, and Data Constraints}
% 	\end{center}
% \end{figure}

% \section{Development Plan}

% \plt{This section is optional.  It is used to explain the plan for developing
%   the software.  In particular, this section gives a list of the order in which
%   the requirements will be implemented.  In the context of a course  this is
%   where you can indicate which requirements will be implemented as part of the
%   course, and which will be ``faked'' as future work.  This section can be
%   organized as a prioritized list of requirements, or it could should the
%   requirements that will be implemented for ``phase 1'', ``phase 2'', etc.}

\section{Values of Auxiliary Constants}

\plt{Show the values of the symbolic parameters introduced in the report.}

\plt{The definition of the requirements will likely call for SYMBOLIC\_CONSTANTS.
Their values are defined in this section for easy maintenance.}

\plt{The value of FRACTION, for the Maintainability NFR would be given here.}

\newpage

\bibliographystyle{plainnat}
\bibliography{mybib}

\end{document}