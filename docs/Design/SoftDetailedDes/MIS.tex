\documentclass[12pt, titlepage]{article}

\usepackage{amsmath, mathtools}

\usepackage[round]{natbib}
\usepackage{amsfonts}
\usepackage{amssymb}
\usepackage{graphicx}
\usepackage{colortbl}
\usepackage{xr}
\usepackage{hyperref}
\usepackage{longtable}
\usepackage{xfrac}
\usepackage{tabularx}
\usepackage{float}
\usepackage{siunitx}
\usepackage{booktabs}
\usepackage{multirow}
\usepackage[section]{placeins}
\usepackage{caption}
\usepackage{fullpage}

\hypersetup{
bookmarks=true,     % show bookmarks bar?
colorlinks=true,       % false: boxed links; true: colored links
linkcolor=red,          % color of internal links (change box color with linkbordercolor)
citecolor=blue,      % color of links to bibliography
filecolor=magenta,  % color of file links
urlcolor=cyan          % color of external links
}

\usepackage{array}

\externaldocument{../../SRS/SRS}

%% Comments

\usepackage{color}

\newif\ifcomments\commentstrue %displays comments
%\newif\ifcomments\commentsfalse %so that comments do not display

\ifcomments
\newcommand{\authornote}[3]{\textcolor{#1}{[#3 ---#2]}}
\newcommand{\todo}[1]{\textcolor{red}{[TODO: #1]}}
\else
\newcommand{\authornote}[3]{}
\newcommand{\todo}[1]{}
\fi

\newcommand{\wss}[1]{\authornote{blue}{SS}{#1}} 
\newcommand{\plt}[1]{\authornote{magenta}{TPLT}{#1}} %For explanation of the template
\newcommand{\an}[1]{\authornote{cyan}{Author}{#1}}

%% Common Parts

\newcommand{\progname}{2D Localizer} % PUT YOUR PROGRAM NAME HERE
\newcommand{\authname}{Aliyah Jimoh} % AUTHOR NAMES                  

\usepackage{hyperref}
    \hypersetup{colorlinks=true, linkcolor=blue, citecolor=blue, filecolor=blue,
                urlcolor=blue, unicode=false}
    \urlstyle{same}
                                


\begin{document}

\title{Module Interface Specification for \progname{}}

\author{\authname}

\date{\today}

\maketitle

\pagenumbering{roman}

\section{Revision History}

\begin{tabularx}{\textwidth}{p{3cm}p{2cm}X}
\toprule {\bf Date} & {\bf Version} & {\bf Notes}\\
\midrule
2025/03/19 & 1.0 & Initial Draft\\
% Date 2 & 1.1 & Notes\\
\bottomrule
\end{tabularx}

~\newpage

\section{Symbols, Abbreviations and Acronyms}

See SRS Documentation at \url{https://github.com/AliyahJimoh/2D-Localizer/blob/main/docs/SRS/SRS.pdf}

\wss{Also add any additional symbols, abbreviations or acronyms}

\newpage

\tableofcontents

\newpage

\pagenumbering{arabic}

\section{Introduction}

The following document details the Module Interface Specifications for~\progname, a program that implements various sensors to help localize mobile robots on a 2D plane in enclosed environments.

Complementary documents include the System Requirement Specifications
and Module Guide.  The full documentation and implementation can be
found at \url{https://github.com/AliyahJimoh/2D-Localizer}.

\section{Notation}

\wss{You should describe your notation.  You can use what is below as
  a starting point.}

The structure of the MIS for modules comes from \citet{HoffmanAndStrooper1995},
with the addition that template modules have been adapted from
\cite{GhezziEtAl2003}.  The mathematical notation comes from Chapter 3 of
\citet{HoffmanAndStrooper1995}.  For instance, the symbol := is used for a
multiple assignment statement and conditional rules follow the form $(c_1
\Rightarrow r_1 | c_2 \Rightarrow r_2 | ... | c_n \Rightarrow r_n )$.

The following table summarizes the primitive data types used by \progname. 

\begin{center}
\renewcommand{\arraystretch}{1.2}
\noindent 
\begin{tabular}{l l p{7.5cm}} 
\toprule 
\textbf{Data Type} & \textbf{Notation} & \textbf{Description}\\ 
\midrule
character & char & a single symbol or digit\\
integer & $\mathbb{Z}$ & a number without a fractional component in (-$\infty$, $\infty$) \\
natural number & $\mathbb{N}$ & a number without a fractional component in [1, $\infty$) \\
real & $\mathbb{R}$ & any number in (-$\infty$, $\infty$)\\
\bottomrule
\end{tabular} 
\end{center}

\noindent
The specification of \progname \ uses some derived data types: sequences, strings, and
tuples. Sequences are lists filled with elements of the same data type. Strings
are sequences of characters. Tuples contain a list of values, potentially of
different types. In addition, \progname \ uses functions, which
are defined by the data types of their inputs and outputs. Local functions are
described by giving their type signature followed by their specification.

\section{Module Decomposition}

The following table is taken directly from the Module Guide document for this project.

\begin{table}[h!]
  \centering
  \begin{tabular}{p{0.3\textwidth} p{0.6\textwidth}}
  \toprule
  \textbf{Level 1} & \textbf{Level 2}\\
  \midrule
  
  {Hardware-Hiding Module} & ~ \\
  \midrule
  
  \multirow{7}{0.3\textwidth}{Behaviour-Hiding Module} & GTSAM Module \\
  & Input Format Module\\
  & Output Module\\
  & Localization Module\\
  & Control Module\\
  & Accuracy Evaluation Module\\
  \midrule
  
  {Software Decision Module} & Plotting Module \\
  
  \bottomrule
  
  \end{tabular}
  \caption{Module Hierarchy}
  \label{TblMH}
  \end{table}

\newpage
~\newpage

\section{MIS of Control Module} \label{M_Control} 
% \wss{Use labels for cross-referencing}

% \wss{You can reference SRS labels, such as R\ref{R_In_position}.}

% \wss{It is also possible to use \LaTeX for hypperlinks to external documents.}

\subsection{Module}

main

\subsection{Uses}
\begin{itemize}
  \item Input Format Module (Section \ref{M_Input})
  \item Localization Module (Section \ref{M_Localize})
  \item Accuracy Evaluation Module (Section \ref{M_Accurate})
  \item Plotting Module (Section \ref{M_Plotting})
  \item Output Module (Section \ref{M_Output})
\end{itemize}


\subsubsection{Exported Constants}
None

\subsubsection{Exported Access Programs}

\begin{center}
\begin{tabular}{p{2cm} p{4cm} p{4cm} p{2cm}}
\hline
\textbf{Name} & \textbf{In} & \textbf{Out} & \textbf{Exceptions} \\
\hline
main& - & - & - \\
\hline
\end{tabular}
\end{center}

\subsection{Semantics}

\subsubsection{State Variables}
None

\subsubsection{Environment Variables}
\begin{itemize}
  \item data\_queue: sequence of tuples \( Q[t] \) where \( Q[t] \) is the update at time \( t \) in the multiprocessing queue 
\end{itemize}

\subsubsection{Assumptions}

% \begin{itemize}
  
% \end{itemize}
\subsubsection{Access Routine Semantics}

\noindent main():
\begin{itemize}
\item transition: Modifying data\_queue with each iteration of range measurements as the Plotting and Output modules get updated 
% \item output: The control module is the first to be called 
% \item exception: None
\end{itemize}

\noindent \textit{\# Get Data}\\\\
\noindent input $=$ InputData()\\\\
\noindent \textit{\# Start the Output Data}\\\\
\noindent data\_queue $=$ Queue()\\\\
\noindent process $=$ Process(target=run\_gui, args$=$(data\_queue,))\\\\
\noindent process.start()\\\\
\noindent m = np.size($\mathbf{\tilde{d}}$, 0)\\\\
\noindent \textit{\# Getting estimated pose for each set of measurements}\\\\
\noindent for t in range(1,m): \\\\
\indent $\mathbf{\hat{x}}$:= localize($\mathbf{a}$, $T_{mf}$, $T_{rf}$, $\mathbf{\tilde{d}}[t,:]$)\\\\
\indent \textit{\# Computing FIM \& CRLB} \\\\
\indent  fim = compute\_fim($\mathbf{\hat{x}}$, $\mathbf{a}$, variances($\boldsymbol{\sigma^2}$))\\\\
\indent  crlb = compute\_crlb(fim) \textit{\# Will be printed}\\\\
\indent update\_trajectory($\mathbf{\hat{x}}$)\\\\
\indent data\_queue.put((t, $\mathbf{\hat{x}}$.x(), $\mathbf{\hat{x}}$.y(), $\mathbf{\hat{x}}$.theta()))\\\\
\noindent \textit{\# Plot on the map}\\\\
\noindent plot\_localization\_live($\mathbf{a}$, $T_{mf}$, map)

% \noindent Get(user\textunderscore input) variables from user\\\\
% \noindent load\textunderscore input(user\textunderscore input)\\\\
% \noindent \textit{\# verify\textunderscore input(All inputs from file)}\\\\
% \noindent )\\\\
% \noindent \textit{\# plot(estimated poses, beacons, $T_{mf}$)}\\\\




% \wss{A module without environment variables or state variables is unlikely to
%   have a state transition.  In this case a state transition can only occur if
%   the module is changing the state of another module.}

% \wss{Modules rarely have both a transition and an output.  In most cases you
%   will have one or the other.}

% \subsubsection{Local Functions}

% \wss{As appropriate} \wss{These functions are for the purpose of specification.
%   They are not necessarily something that is going to be implemented
%   explicitly.  Even if they are implemented, they are not exported; they only
%   have local scope.}

\newpage

\section{MIS of GTSAM Module} \label{M_GTSAM} 
% \wss{Use labels for
%   cross-referencing}

% \wss{You can reference SRS labels, such as R\ref{R_In_position}.}

% \wss{It is also possible to use \LaTeX for hypperlinks to external documents.}

\subsection{Module}

gtsam\_wrapper

\subsection{Uses}
None

\subsection{Syntax}

\subsubsection{Exported Constants}
None
\subsubsection{Exported Access Programs}

\begin{center}
\begin{tabular}{p{6cm} p{6cm} p{2cm} p{3cm}}
\hline
\textbf{Name} & \textbf{In} & \textbf{Out} & \textbf{Exceptions} \\
\hline
Pose2 & $x: \mathbb{R}, y: \mathbb{R}, \theta: \mathbb{R}$ & $\mathbb{R}^{3}$ & - \\
Point2 & $x: \mathbb{R}, y: \mathbb{R}$  & $\mathbb{R}^{2}$  & - \\
symbol & char: char, int: $\mathbb{Z}$  & - & - \\
NonlinearFactorGraph & - & Graph & - \\
PriorFactorPose2 & $key:\mathbb{Z}, \textbf{pose}: \mathbb{R}^3, noise: Model$ & Factor & - \\
PriorFactorPoint2 & $key:\mathbb{Z}, \textbf{pose}: \mathbb{R}^2, noise: Model$& Factor& - \\
RangeFactor2D &  $key1: \mathbb{Z}, key2: \mathbb{Z}, d: \mathbb{R}, noise: Model$  & Factor & - \\
noiseModel\_Isotropic\_Sigma & $dim: \mathbb{Z}, \sigma: \mathbb{R}$ & Model & - \\
LevenbergMarquardtOptimizer & $graph: Graph, values: Values$ & Values & - \\
Values & - & Values & - \\
insert & $values: Values, key: \mathbb{Z}, value: Any$ & - & - \\
atPose2 & $result: Values, key: \mathbb{Z}$  & $\mathbb{R}^3$ & - \\
compose & $T_1: \mathbb{R}^3, T_2: \mathbb{R}^3$ & $\mathbb{R}^3$ & - \\
inverse &  $pose: \mathbb{R}^3$ & $\mathbb{R}^3$ & - \\
\hline
\end{tabular}
\end{center}

\subsection{Semantics}

\subsubsection{State Variables}
None

\subsubsection{Environment Variables}
None

\subsubsection{Assumptions}
\begin{itemize}
  \item The module will call on a yaml file
\end{itemize}

\subsubsection{Access Routine Semantics}

Pose2($x, y, \theta$):
\begin{itemize}
    \item \textbf{output}: $ \mathbb{R}^3 $ (A 2D pose with orientation)
\end{itemize}

\noindent Point2($x, y$):
\begin{itemize}
    \item \textbf{output}: $ \mathbb{R}^2 $ (A 2D point)
\end{itemize}

\noindent symbol($char, int$):
\begin{itemize}
    \item \textbf{output}: $char$ (A GTSAM symbol key)
\end{itemize}

\noindent NonlinearFactorGraph():
\begin{itemize}
    \item \textbf{output}: Graph (An empty nonlinear factor graph)
\end{itemize}

\noindent PriorFactorPose2($key, pose, noise\_model$):
\begin{itemize}
    \item \textbf{output}: Factor (A prior factor on a 2D pose)
\end{itemize}

\noindent PriorFactorPoint2($key, point, noise\_model$):
\begin{itemize}
    \item \textbf{output}: Factor (A prior factor on a 2D point)
\end{itemize}

\noindent RangeFactor2D($key_1, key_2, measured, noise\_model$):
\begin{itemize}
    \item \textbf{output}: Factor (A range factor between two keys)
\end{itemize}

\noindent noiseModel\_Isotropic\_Sigma($dim, \sigma$):
\begin{itemize}
    \item \textbf{output}: Model (An isotropic noise model)
\end{itemize}

\noindent LevenbergMarquardtOptimizer($graph, values$):
\begin{itemize}
    \item \textbf{output}: Values (Optimized results from factor graph)
\end{itemize}

\noindent Values():
\begin{itemize}
    \item \textbf{output}: Values (An empty values container)
\end{itemize}

\noindent insert($values, key, value$):
\begin{itemize}
    \item \textbf{output}: \textit{None} (Modifies values in place)
\end{itemize}

\noindent atPose2($result, key$):
\begin{itemize}
    \item \textbf{output}: $\mathbb{R}^3$ (The retrieved pose from results)
\end{itemize}

\noindent compose($T_1, T_2$):
\begin{itemize}
    \item \textbf{output}: $\mathbb{R}^3$ (The composition of two poses)
\end{itemize}

\noindent inverse($T$):
\begin{itemize}
    \item \textbf{output}: $\mathbb{R}^3$ (The inverse of a pose)
\end{itemize}

% \wss{A module without environment variables or state variables is unlikely to
%   have a state transition.  In this case a state transition can only occur if
%   the module is changing the state of another module.}

% \wss{Modules rarely have both a transition and an output.  In most cases you
%   will have one or the other.}

\newpage

\section{MIS of Input Format Module} \label{M_Input} 
% \wss{Use labels for
%   cross-referencing}

% \wss{You can reference SRS labels, such as R\ref{R_In_position}.}

% \wss{It is also possible to use \LaTeX for hypperlinks to external documents.}

\subsection{Module}

input\_format

\subsection{Uses}
\begin{itemize}
  \item GTSAM Module (Section \ref{M_GTSAM})
\end{itemize}

\subsection{Syntax}

\subsubsection{Exported Constants}
None
\subsubsection{Exported Access Programs}

\begin{center}
\begin{tabular}{p{3cm} p{2cm} p{2cm} p{4cm}}
\hline
\textbf{Name} & \textbf{In} & \textbf{Out} & \textbf{Exceptions} \\
\hline
load\_input & -& -& FileNotFoundError \\
 &  &  & ValueError \\
get\_beacons& - & $\mathbb{R}^{N \times 2}$ & - \\
get\_fmMap& - & $\mathbb{R}^3$ & - \\
get\_fmRobots& - & $\mathbb{R}^3$ & - \\
get\_map& - & String & - \\
get\_ranges& - & $\mathbb{R}^N$ & - \\
get\_variances& - & $\mathbb{R}^N$ & - \\
\hline
\end{tabular}
\end{center}

\subsection{Semantics}

\subsubsection{State Variables}
\begin{itemize}
  \item sensor\textunderscore data(user\textunderscore input): 
  \begin{itemize}
    \item range\textunderscore measurements: $\mathbb{R}^N$
    \item camera: $\mathbb{R}^3$
    \item variances: $\mathbb{R}^N$
  \end{itemize}
\end{itemize}

\subsubsection{Environment Variables}
None

\subsubsection{Assumptions}
\begin{itemize}
  \item The module will call on a yaml file
\end{itemize}

\subsubsection{Access Routine Semantics}

\noindent load\_input():
\begin{itemize}
    \item output: None
    \item exception: \textit{FileNotFoundError}, \textit{ValueError}
\end{itemize}

\noindent input.get\_beacons():
\begin{itemize}
    \item output:$=\mathbf{a}$
    \item exception: None
\end{itemize}

\noindent get\_fmMap():
\begin{itemize}
    \item output: $T_{mf}=Pose2(\mathbb{R}^{3})$
    \item exception: None
\end{itemize}

\noindent get\_fmRobot():
\begin{itemize}
    \item output: $T_{rf}=Pose2(\mathbb{R}^{3})$
    \item exception: None
\end{itemize}

\noindent get\_map():
\begin{itemize}
    \item output:$=$ \lq Image.png\rq
    \item exception: None
\end{itemize}

\noindent get\_ranges():
\begin{itemize}
    \item output:$=\mathbf{\tilde{d}}$
    \item exception: None
\end{itemize}

\noindent get\_variances():
\begin{itemize}
    \item output: $\boldsymbol{\sigma^2}$
    \item exception: None
\end{itemize}
% \wss{A module without environment variables or state variables is unlikely to
%   have a state transition.  In this case a state transition can only occur if
%   the module is changing the state of another module.}

% \wss{Modules rarely have both a transition and an output.  In most cases you
%   will have one or the other.}

\newpage

\section{MIS of Localization Module} \label{M_Localize}

\subsection{Module}

localization

\subsection{Uses}
\begin{itemize}
  \item Input Format Module (Section \ref{M_Input})
  
\end{itemize}
\subsection{Syntax}

\subsubsection{Exported Constants}
None

\subsubsection{Exported Access Programs}

\begin{center}
\begin{tabular}{p{2cm} p{4cm} p{4cm} p{2cm}}
\hline
\textbf{Name} & \textbf{In} & \textbf{Out} & \textbf{Exceptions} \\
\hline
localize & User Data & $\mathbb{R}^3$ & - \\
\hline
\end{tabular}
\end{center}

\subsection{Semantics}

\subsubsection{State Variables}
\textit{initial/current pose}

\subsubsection{Environment Variables}
None

\subsubsection{Assumptions}
\begin{itemize}
  \item GTSAM is installed
\end{itemize}

\subsubsection{Access Routine Semantics}

\noindent localize(beacons, fm\textunderscore map, fm\textunderscore robot, range\textunderscore m):
\begin{itemize}
% \item transition: \wss{if appropriate} 
\item output: Estimated pose of the robot
\begin{itemize}
  \item estimated\textunderscore pose: $\mathbb{R}^3$ 
\end{itemize}
\item exception: \textit{Format errors}
\end{itemize}

\subsubsection{Local Functions}
None
\newpage

\section{MIS of Accuracy Evaluation Module} \label{M_Accurate}

\subsection{Module}

accuracy

\subsection{Uses}
\begin{itemize}
  \item Localization Module (Section \ref{M_Localize})
\end{itemize}


\subsection{Syntax}

\subsubsection{Exported Constants}
None

\subsubsection{Exported Access Programs}

\begin{center}
\begin{tabular}{p{4cm} p{4cm} p{4cm} p{2cm}}
\hline
\textbf{Name} & \textbf{In} & \textbf{Out} & \textbf{Exceptions} \\
\hline
compute\textunderscore fim & ($\mathbb{R}^2, \mathbb{R}^{N \times 2}, \mathbb{R}^N$)  & $\mathbb{R}^{2 \times 2}$ & - \\
compute\textunderscore crlb & $\mathbb{R}^{2 \times 2}$ & $\mathbb{R}^{2 \times 2}$ & - \\
\hline
\end{tabular}
\end{center}

\subsection{Semantics}

\subsubsection{State Variables}
None

\subsubsection{Environment Variables}
None

\subsubsection{Assumptions}
\begin{itemize}
  \item Noise variances are positive
\end{itemize}

\subsubsection{Access Routine Semantics}

\noindent compute\_fim(estimated\textunderscore pose, beacons, range\textunderscore variances):
\begin{itemize}
% \item transition: \wss{if appropriate} 
\item output: A \( 2 \times 2 \) Fisher Information Matrix (FIM), computed as:
\[
\mathcal{I}(\hat{\mathbf{x}}) = \sum_{j=1}^{N} \frac{1}{\sigma_j^2} \frac{(\hat{\mathbf{x}} - \mathbf{a}_j)(\hat{\mathbf{x}} - \mathbf{a}_j)^T}{\|\hat{\mathbf{x}} - \mathbf{a}_j\|^2}
\]
\end{itemize}

\noindent compute\_fim(estimated\textunderscore pose, beacons, range\textunderscore variances):
\begin{itemize}
% \item transition: \wss{if appropriate} 
\item output: A \( 2 \times 2 \) CRLB matrix, computed as:
\[
\mathcal{C} = \mathcal{I}^{-1}
\]
\item exception: \textit{If \( \mathcal{I} \) is singular, the function returns `None`.}
\end{itemize}

\subsubsection{Local Functions}
None

\newpage

\section{MIS of Output Module} \label{M_Output} 
\subsection{Module}

output

\subsection{Uses}
\begin{itemize}
  \item Localization Module (Section \ref{M_Localize})
\end{itemize}

\subsection{Syntax}

\subsubsection{Exported Constants}
None

\subsubsection{Exported Access Programs}

\begin{center}
\begin{tabular}{p{4cm} p{4cm} p{4cm} p{2cm}}
\hline
\textbf{Name} & \textbf{In} & \textbf{Out} & \textbf{Exceptions} \\
\hline
output\_format & - & - & - \\
output\_pose & - & - & - \\
\hline
\end{tabular}
\end{center}

\subsection{Semantics}

\subsubsection{State Variables}
None
\subsubsection{Environment Variables}
None

\subsubsection{Assumptions}
\begin{itemize}
  \item 
\end{itemize}
\subsubsection{Access Routine Semantics}

\noindent output\_format():
\begin{itemize}
\item transition: \wss{if appropriate} 
\item output: \wss{if appropriate} 
\item exception: \wss{if appropriate} 
\end{itemize}

\noindent output\_pose():
\begin{itemize}
\item transition: \wss{if appropriate} 
\item output: \wss{if appropriate} 
\item exception: \wss{if appropriate} 
\end{itemize}


\subsubsection{Local Functions}
None

\newpage

\section{MIS of Plotting Module} \label{M_Plotting}

\subsection{Module}

plot 

\subsection{Uses}


\subsection{Syntax}

\subsubsection{Exported Constants}

\subsubsection{Exported Access Programs}

\begin{center}
\begin{tabular}{p{4cm} p{4cm} p{4cm} p{2cm}}
\hline
\textbf{Name} & \textbf{In} & \textbf{Out} & \textbf{Exceptions} \\
\hline
plot\_localization\_live & \( R^{N \times 2}, R^2, \text{Image} \) & Plot & - \\
\hline
update\_trajectory & \( R^3 \) & - & - \\
\hline
\end{tabular}
\end{center}

\subsection{Semantics}

\subsubsection{State Variables}
None

\subsubsection{Environment Variables}
None

\subsubsection{Assumptions}
None

\subsubsection{Access Routine Semantics}

\noindent plot\_localization\_live(beacons, fm\_map\_2D, map):
\begin{itemize}
\item output: A dynamic plot showing real-time robot localization.
\end{itemize}

\noindent update\_trajectory(estimated\_pose):
\begin{itemize}
\item transition: Changes the robot's position on the map
\end{itemize}


\subsubsection{Local Functions}
None

\newpage


\bibliographystyle {plainnat}
\bibliography {../../../refs/References}

\newpage

\section{Appendix} \label{Appendix}

\wss{Extra information if required}

\end{document}