\documentclass[12pt, titlepage]{article}

\usepackage{amsmath, mathtools}

\usepackage[round]{natbib}
\usepackage{amsfonts}
\usepackage{amssymb}
\usepackage{graphicx}
\usepackage{colortbl}
\usepackage{xr}
\usepackage{hyperref}
\usepackage{longtable}
\usepackage{xfrac}
\usepackage{tabularx}
\usepackage{float}
\usepackage{siunitx}
\usepackage{booktabs}
\usepackage{multirow}
\usepackage[section]{placeins}
\usepackage{caption}
\usepackage{fullpage}

\hypersetup{
bookmarks=true,     % show bookmarks bar?
colorlinks=true,       % false: boxed links; true: colored links
linkcolor=red,          % color of internal links (change box color with linkbordercolor)
citecolor=blue,      % color of links to bibliography
filecolor=magenta,  % color of file links
urlcolor=cyan          % color of external links
}

\usepackage{array}

\externaldocument{../../SRS/SRS}

%% Comments

\usepackage{color}

\newif\ifcomments\commentstrue %displays comments
%\newif\ifcomments\commentsfalse %so that comments do not display

\ifcomments
\newcommand{\authornote}[3]{\textcolor{#1}{[#3 ---#2]}}
\newcommand{\todo}[1]{\textcolor{red}{[TODO: #1]}}
\else
\newcommand{\authornote}[3]{}
\newcommand{\todo}[1]{}
\fi

\newcommand{\wss}[1]{\authornote{blue}{SS}{#1}} 
\newcommand{\plt}[1]{\authornote{magenta}{TPLT}{#1}} %For explanation of the template
\newcommand{\an}[1]{\authornote{cyan}{Author}{#1}}

%% Common Parts

\newcommand{\progname}{2D Localizer} % PUT YOUR PROGRAM NAME HERE
\newcommand{\authname}{Aliyah Jimoh} % AUTHOR NAMES                  

\usepackage{hyperref}
    \hypersetup{colorlinks=true, linkcolor=blue, citecolor=blue, filecolor=blue,
                urlcolor=blue, unicode=false}
    \urlstyle{same}
                                


\begin{document}

\title{Module Interface Specification for \progname{}}

\author{\authname}

\date{\today}

\maketitle

\pagenumbering{roman}

\section{Revision History}

\begin{tabularx}{\textwidth}{p{3cm}p{2cm}X}
\toprule {\bf Date} & {\bf Version} & {\bf Notes}\\
\midrule
2025/03/19 & 1.0 & Initial Draft\\
2025/04/18 & 1.1 & Implement Feedback\\
\bottomrule
\end{tabularx}

~\newpage

\section{Symbols, Abbreviations and Acronyms}

See SRS Documentation at \url{https://github.com/AliyahJimoh/2D-Localizer/blob/main/docs/SRS/SRS.pdf}. The symbols used in this document are mentioned below.

\begin{table}[H]
  \centering
  \begin{tabular}{l l}
    \hline
    \textbf{Symbol} & \textbf{Description} \\
    \hline
    \progname & 2D Localization Solution \\
    CRLB & Cramer-Rao Lower Bound \\
    FIM & Fisher Information Matrix\\
    FM & Fiducial Markers\\
    map & String of the map image's name \\
    $N$ & Number of beacons used \\
    $M$ & Number of FMs used \\
    $p$ & Number of positions the robot has in its trajectory \\
    $\mathbf{path}$ & $\mathbb{R}^{p \times 3}$ matrix of the ground truth trajectory\\
    $\mathbf{a}$ & $\mathbb{R}^{N \times 2}$ matrix of beacon coordinates \\
    $\mathbf{C}$ & $\mathbb{R}^{2 \times2}$ CRLB \\
    $\boldsymbol{\mathcal{I}}(\mathbf{\hat{x}})$ & $\mathbb{R}^{2 \times 2}$ FIM\\
    $\mathbf{T}_{mf}$ & $\mathbb{R}^{M \times 3}$ pose of fiducial marker in map frame ($x, y, \theta$) \\
    $\mathbf{T}_{rf}$ & $\mathbb{R}^3$ pose of fiducial marker in robot frame ($x, y, \theta$) \\
    $\tilde{\mathbf{d}}$ & $\mathbb{R}^N$ vector of a set of range measurements \\
    $\tilde{\mathbf{D}}$ & $\mathbb{R}^{p \times N}$ matrix of range measurements in all positions \\
    $\mathbf{\hat{x}}$ & $\mathbb{R}^3$ estimated robot pose \\
    $i$ & Index of fiducial marker \\
    $j$ & Index of beacon \\
    $t$ & Index of robot position along the trajectory \\
    $x$ & $\mathbb{R}$ x coordinate of robot \\
    $y$ & $\mathbb{R}$ y coordinate of robot \\
    $\theta$ & $\mathbb{R}$ orientation of robot (radians) \\
    $\eta$ & Gaussian sensor noise \\
    $\phi$ & Angle of FM relative to robot (used for FOV filtering) \\
    $\boldsymbol{\sigma^2}$ & $\mathbb{R}^N$ vector of range noise variances \\
    \hline
  \end{tabular}
  \caption{Symbol Definitions Used in Access Routines}
  \end{table}
  


\newpage

\tableofcontents

\newpage

\pagenumbering{arabic}

\section{Introduction}

The following document details the Module Interface Specifications for~\progname, a program that implements various sensors to help localize mobile robots on a 2D plane in enclosed environments.

Complementary documents include the System Requirement Specifications
and Module Guide.  The full documentation and implementation can be
found at \url{https://github.com/AliyahJimoh/2D-Localizer}.

\section{Notation}

The structure of the MIS for modules comes from \citet{HoffmanAndStrooper1995},
with the addition that template modules have been adapted from
\cite{GhezziEtAl2003}.  The mathematical notation comes from Chapter 3 of
\citet{HoffmanAndStrooper1995}.  For instance, the symbol := is used for a
multiple assignment statement and conditional rules follow the form $(c_1
\Rightarrow r_1 | c_2 \Rightarrow r_2 | ... | c_n \Rightarrow r_n )$.

The following table summarizes the primitive data types used by \progname. 

\begin{center}
\renewcommand{\arraystretch}{1.2}
\noindent 
\begin{tabular}{l l p{7.5cm}} 
\toprule 
\textbf{Data Type} & \textbf{Notation} & \textbf{Description}\\ 
\midrule
boolean & $\mathbb{B}$ & a true or false value {True, False}  \\
character & char & a single symbol or digit\\
dictionary & dict & a key-value data structure where keys map to values of different types\\
integer & $\mathbb{Z}$ & a number without a fractional component in (-$\infty$, $\infty$) \\
list & list & An ordered collection of items of different or similar types \\
real & $\mathbb{R}$ & any number in (-$\infty$, $\infty$)\\
string & String & more than one symbol put together\\
\bottomrule
\end{tabular} 
\end{center}

\noindent
The specification of \progname \ uses some derived data types: sequences, strings, and
tuples. Sequences are lists filled with elements of the same data type. Strings
are sequences of characters. Tuples contain a list of values, potentially of
different types. In addition, \progname \ uses functions, which
are defined by the data types of their inputs and outputs. Local functions are
described by giving their type signature followed by their specification.

\section{Module Decomposition}

The following table is taken directly from the Module Guide document for this project.

\begin{table}[h!]
  \centering
  \begin{tabular}{p{0.3\textwidth} p{0.6\textwidth}}
  \toprule
  \textbf{Level 1} & \textbf{Level 2}\\
  \midrule
  
  {Hardware-Hiding Module} & ~ \\
  \midrule
  
  \multirow{7}{0.3\textwidth}{Behaviour-Hiding Module} & Input Format Module \\
  & Simulation Module\\
  & Output Module\\
  & Localization Module\\
  & Control Module\\
  & Accuracy Evaluation Module\\
  \midrule
  
  {Software Decision Module} & GTSAM Module \\
  & Plotting Module \\
 
  \bottomrule
  
  \end{tabular}
  \caption{Module Hierarchy}
  \label{TblMH}
  \end{table}

~\newpage

\section{MIS of Control Module} \label{M_Control} 
% \wss{Use labels for cross-referencing}

% \wss{You can reference SRS labels, such as R\ref{R_In_position}.}

% \wss{It is also possible to use \LaTeX for hypperlinks to external documents.}

\subsection{Module}

main

\subsection{Uses}
\begin{itemize}
  \item Input Format Module (Section \ref{M_Input})
  \item Localization Module (Section \ref{M_Localize})
  \item Accuracy Evaluation Module (Section \ref{M_Accurate})
  \item Plotting Module (Section \ref{M_Plotting})
  \item Output Module (Section \ref{M_Output})
  \item Python multiprocessing Library (Queue, Process)
\end{itemize}


\subsubsection{Exported Constants}
None

\subsubsection{Exported Access Programs}

\begin{center}
\begin{tabular}{p{2cm} p{4cm} p{4cm} p{2cm}}
\hline
\textbf{Name} & \textbf{In} & \textbf{Out} & \textbf{Exceptions} \\
\hline
main& - & - & - \\
\hline
\end{tabular}
\end{center}

\subsection{Semantics}

\subsubsection{State Variables}
\begin{itemize}
  \item data\_queue: Queue that stores the estimated pose of the current position and passes it to the Output Module's environmental variable.
\end{itemize}

\subsubsection{Environment Variables}
None

\subsubsection{Assumptions}
None
% \begin{itemize}
  
% \end{itemize}
\subsubsection{Access Routine Semantics} \label{main_code}

\noindent main():
\begin{itemize}
\item transition: Modifies the shared queue (`data\_queue`) with each set of range and camera measurements.
% \item output: The control module is the first to be called 
% \item exception: None
\end{itemize}

\noindent \textit{\# Get Data}\\\\
\noindent input $=$ InputData() \textit{\# Abstract Data Type from Input Format Module}\\\\
\noindent \textit{\# Start the Output Data}\\\\
\noindent data\_queue $=$ Queue() \textit{\# Part of multiprocessing library}\\\\
\noindent process $=$ Process(target=run\_gui, args$=$[data\_queue]) \textit{\# Runs the GUI (Output Module). It is created with run\_gui as the target function and data\_queue as its input argument.}\\\\
\noindent process.start()\\\\
\noindent m = p \textit{\# Number of positions the robot has in the map}\\\\
\noindent \textit{\# Getting estimated pose for each set of measurements}\\\\
\noindent for t in range(1,m): \\\\
\indent $\mathbf{\hat{x}}$:= localize($\mathbf{a}$, $T_{mf}$, $\mathbf{\tilde{D}}[t,:]$, $\mathbf{path}[t,:]$)\\\\
\indent \textit{\# Computing FIM \& CRLB} \\\\
\indent  fim = compute\_fim($\mathbf{\hat{x}}$, $\mathbf{a}$, variances($\boldsymbol{\sigma^2}$))\\\\
\indent  crlb = compute\_crlb(fim) \textit{\# Will be printed}\\\\
\indent update\_trajectory($\mathbf{\hat{x}}$)\\\\
\indent data\_queue.put((t, $\mathbf{\hat{x}}$.x(), $\mathbf{\hat{x}}$.y(), $\mathbf{\hat{x}}$.theta()))\\\\
\noindent \textit{\# Plot on the map}\\\\
\noindent plot\_localization\_live($\mathbf{a}$, $T_{mf}$, map, $\mathbf{path}$, map\_size)

% \noindent Get(user\textunderscore input) variables from user\\\\
% \noindent load\textunderscore input(user\textunderscore input)\\\\
% \noindent \textit{\# verify\textunderscore input(All inputs from file)}\\\\
% \noindent )\\\\
% \noindent \textit{\# plot(estimated poses, beacons, $T_{mf}$)}\\\\




% \wss{A module without environment variables or state variables is unlikely to
%   have a state transition.  In this case a state transition can only occur if
%   the module is changing the state of another module.}

% \wss{Modules rarely have both a transition and an output.  In most cases you
%   will have one or the other.}

% \subsubsection{Local Functions}

% \wss{As appropriate} \wss{These functions are for the purpose of specification.
%   They are not necessarily something that is going to be implemented
%   explicitly.  Even if they are implemented, they are not exported; they only
%   have local scope.}

\newpage

\section{MIS of GTSAM Module} \label{M_GTSAM} 
% \wss{Use labels for
%   cross-referencing}

% \wss{You can reference SRS labels, such as R\ref{R_In_position}.}

% \wss{It is also possible to use \LaTeX for hypperlinks to external documents.}
The GTSAM Module provides wrapper access the Georgia Tech Smoothing and Mapping (GTSAM) library using simplified and consistent Python interfaces. GTSAM is used for solving estimation problems using factor graphs which are a way to represent relationships between variables using "factors"(pieces of information gotten from sensors). More information about the API can be found at \url{https://gtsam.org/doxygen/index.html}

\subsection{Module}

gtsam\_wrapper

\subsection{Uses}
None

\subsection{Syntax}

\subsubsection{Exported Constants}
None
\subsubsection{Exported Types}

\begin{center}
  \renewcommand{\arraystretch}{1.2}
  \noindent 
  \begin{tabular}{l l p{7.5cm}} 
  \toprule 
  \textbf{Data Type} & \textbf{Notation} & \textbf{Description}\\ 
  \midrule
  factor & Factor & a constraint in a factor graph that relates variables\\
  factor graph & Graph & a collection of factors defining an optimization problem\\
  noise model & Model & a model that defines uncertainty in a measurement \\
  values & Values & a container that stores variable estimates in a factor graph\\
  \bottomrule
  \end{tabular} 
  \end{center}

\subsubsection{Exported Access Programs}

\begin{center}
\begin{tabular}{p{6cm} p{6cm} p{2cm} p{3cm}}
\hline
\textbf{Name} & \textbf{In} & \textbf{Out} & \textbf{Exceptions} \\
\hline
Pose2 & $x: \mathbb{R}, y: \mathbb{R}, \theta: \mathbb{R}$ & $\mathbb{R}^{3}$ & - \\
Point2 & $x: \mathbb{R}, y: \mathbb{R}$  & $\mathbb{R}^{2}$  & - \\
symbol & char: char, int: $\mathbb{Z}$  & String & - \\
NonlinearFactorGraph & - & Graph & - \\
PriorFactorPose2 & $key:\mathbb{Z}, \textbf{pose}: \mathbb{R}^3, noise: Model$ & Factor & - \\
PriorFactorPoint2 & $key:\mathbb{Z}, \textbf{pose}: \mathbb{R}^2, noise: Model$& Factor& - \\
RangeFactor2D &  $key1: \mathbb{Z}, key2: \mathbb{Z}, d: \mathbb{R}, noise: Model$  & Factor & - \\
BetweenFactor &  $key1: \mathbb{Z}, key2: \mathbb{Z}, between: Pose2, noise: Model$  & Factor & - \\
noiseModel\_Isotropic\_Sigma & $dim: \mathbb{Z}, \sigma: \mathbb{R}$ & Model & - \\
LevenbergMarquardtOptimizer & $graph: Graph, values: Values$ & Values & - \\
Values & - & Values & - \\
insert & $values: Values, key: \mathbb{Z}, value:$ Pose2 or Point2 & - & - \\
atPose2 & $result: Values, key: \mathbb{Z}$  & $\mathbb{R}^3$ & - \\
compose & $T_{mf}: \mathbb{R}^3, T_{rf}: \mathbb{R}^3$ & $\mathbb{R}^3$ & - \\
inverse &  $T_{rf}: \mathbb{R}^3$ & $\mathbb{R}^3$ & - \\
\hline
\end{tabular}
\end{center}

\subsection{Semantics}

\subsubsection{State Variables}
None

\subsubsection{Environment Variables}
None

\subsubsection{Assumptions}
None

\subsubsection{Access Routine Semantics}

Pose2($x, y, \theta$):
\begin{itemize}
    \item output: $out:= [x, y, \theta] $ (A 2D pose with orientation)
    \item exception: None
\end{itemize}

\noindent Point2($x, y$):
\begin{itemize}
    \item output: $out:=  [x, y] $ (2D position)
    \item exception: None
\end{itemize}

\noindent symbol($char, int$):
\begin{itemize}
    \item output: $out:= x1(pose), a1, a2, a3(beacons)$
    \item exception: None
\end{itemize}

\noindent NonlinearFactorGraph():
\begin{itemize}
    \item output: $out:=$ An empty factor graph
    \item exception: None
\end{itemize}

\noindent PriorFactorPose2($key, pose, noise\_model$):
\begin{itemize}
    \item output: $out:=$ Factor (A prior factor on a 2D pose)
    \item exception: None
\end{itemize}

\noindent PriorFactorPoint2($key, point, noise\_model$):
\begin{itemize}
    \item output: $out:=$ Factor (A prior factor on a 2D point)
    \item exception: None
\end{itemize}

\noindent RangeFactor2D($key_1, key_2, measured, noise\_model$):
\begin{itemize}
    \item output: $out:=$ Factor (A range factor between two keys)
    \item exception: None
\end{itemize}

\noindent BetweenFactor($key_1, key_2, measured, noise\_model$):
\begin{itemize}
    \item output: $out:=$ Factor (A factor that enforces a relative transformation between two variables)
    \item exception: None
\end{itemize}

\noindent noiseModel\_Isotropic\_Sigma($dim, \sigma$):
\begin{itemize}
    \item output: $out:=$ Model (An isotropic noise model)
    \item exception: None
\end{itemize}

\noindent LevenbergMarquardtOptimizer($graph, values$):
\begin{itemize}
    \item output: $out:=$ Values (Optimized results from factor graph)
    \item exception: None
\end{itemize}

\noindent Values():
\begin{itemize}
    \item output: $out:=$ Values (An empty values container)
    \item exception: None
\end{itemize}

\noindent insert($Values, key, value$):
\begin{itemize}
    \item transition: Adds point/pose into a Values variable according to its id (key)
    \item exception: None
\end{itemize}

\noindent atPose2($result, key$):
\begin{itemize}
    \item output: $out:= \mathbf{\hat{x}}$ 
    \item exception: None
\end{itemize}

\noindent compose($T_{mf}, T_{rf}$):
\begin{itemize}
    \item output: $out:= T_{mr}$ (The composition of two poses)
    \item exception: None
\end{itemize}

\noindent inverse($T_{rf}$):
\begin{itemize}
    \item output: $out:= T_{fr}$
    \item exception: None
\end{itemize}

\newpage

\section{MIS of Input Format Module} \label{M_Input} 
% \wss{Use labels for
%   cross-referencing}

% \wss{You can reference SRS labels, such as R\ref{R_In_position}.}

% \wss{It is also possible to use \LaTeX for hypperlinks to external documents.}

\subsection{Module}

input\_format

\subsection{Uses}
\begin{itemize}
  \item Simulation Module (Section \ref{M_Simulate})
\end{itemize}

\subsection{Syntax}

\subsubsection{Exported Constants}
None
\subsubsection{Exported Access Programs}

These functions are methods of the `InputData' class instance, which must be initialized before use (example shown in section \ref{main_code}).

\begin{center}
\begin{tabular}{p{3cm} p{2cm} p{2cm} p{4cm}}
\hline
\textbf{Name} & \textbf{In} & \textbf{Out} & \textbf{Exceptions} \\
\hline
load\_input & self & -& FileNotFoundError, ValueError \\
get\_beacons& self & $\mathbb{R}^{N \times 2}$ & - \\
get\_fmMap& self & $\mathbb{R}^3$ & - \\
get\_mapSize& self & $\mathbb{R}^2$ & - \\
get\_trajectory& self & $\mathbb{R}^{p \times 3}$ & - \\
get\_map& self & String & - \\
get\_ranges& self & $\mathbb{R}^N$ & - \\
get\_variances& self & $\mathbb{R}^N$ & - \\
\hline
\end{tabular}
\end{center}

\subsection{Semantics}

\subsubsection{State Variables}
\begin{itemize}
  \item input\_file: String (represents the path to the user input file `user\_input.yaml')
  \item data: dict (storing parsed input data from YAML file).
\end{itemize}

\subsubsection{Environment Variables}
None

\subsubsection{Assumptions}
\begin{itemize}
  \item The module will call on a pre-existing YAML file
\end{itemize}

\subsubsection{Access Routine Semantics}

\noindent load\_input():
\begin{itemize}
  \item transition: Reads the YAML input file and stores it in `self.data`.
    \item exception: FileNotFoundError if the input file is not detected and ValueError if the YAML file is formatted incorrectly
\end{itemize}

\noindent input.get\_beacons():
\begin{itemize}
    \item output: $out:=\mathbf{a}$
    \item exception: None
\end{itemize}

\noindent get\_fmMap():
\begin{itemize}
    \item output: $out:= \mathbb{R}^{M \times 3}$
    \item exception: None
\end{itemize}

\noindent get\_map():
\begin{itemize}
    \item output: out:=  String of picture's name  
    \item exception: None
\end{itemize}

\noindent get\_mapSize():
\begin{itemize}
  \item output: out := [length, width] of map in meters (from user input)
  \item exception: None
\end{itemize}

\noindent get\_trajectory():
\begin{itemize}
  \item output: $out := \mathbb{R}^{p\times 3}$ matrix containing the robot's full trajectory over each position $(x, y, \theta)$
  \item exception: None
\end{itemize}

\noindent get\_ranges():
\begin{itemize}
    \item output: $out:=\mathbf{\tilde{D}}$
    \item exception: None
\end{itemize}

\noindent get\_variances():
\begin{itemize}
    \item output: $out:=\boldsymbol{\sigma^2}$
    \item exception: None
\end{itemize}
% \wss{A module without environment variables or state variables is unlikely to
%   have a state transition.  In this case a state transition can only occur if
%   the module is changing the state of another module.}

% \wss{Modules rarely have both a transition and an output.  In most cases you
%   will have one or the other.}

\newpage

\section{MIS of Simulation Module} \label{M_Simulate} 

% \wss{Use labels for
%   cross-referencing}

% \wss{You can reference SRS labels, such as R\ref{R_In_position}.}

% \wss{It is also possible to use \LaTeX for hypperlinks to external documents.}

\subsection{Module}

simulation

\subsection{Uses}
None

\subsection{Syntax}

\subsubsection{Exported Constants}
None
\subsubsection{Exported Access Programs}

\begin{center}
  \begin{tabular}{p{3cm} p{5.5cm} p{2.5cm} p{3cm}}
  \hline
  \textbf{Name} & \textbf{In} & \textbf{Out} & \textbf{Exceptions} \\
  \hline
  noisy\_range & 
  beacons: $\mathbb{R}^{N \times 2}$, 
  variances: $\mathbb{R}^N$, 
  trajectory: $\mathbb{R}^{p \times 3}$ & 
  $\tilde{d}: \mathbb{R}^{p \times N}$ & 
  None \\
  
  fm\_robot & 
  trajectory: $\mathbb{R}^3$, 
  fm\_map: $\mathbb{R}^3$ & 
  $T_{rf} : \mathbb{R}^3$ & 
  None \\
  
  visible\_fms & 
  robot\_pose:$\mathbb{R}^3$, 
  fm\_map:$\mathbb{R}^{M \times 3}$ & 
  list $(i: \mathbb{Z}, T_{rf}:\mathbb{R}^3)$ & 
  None \\
  \hline
  \end{tabular}
  \end{center}
  

\subsection{Semantics}

\subsubsection{State Variables}
None

\subsubsection{Environment Variables}
None

\subsubsection{Assumptions}
None

\subsubsection{Access Routine Semantics}

\noindent noisy\_range(beacons, variances, trajectory): \\

\noindent \textit{\# extract robot positions} \\
path := trajectory[:, :2] \textit{\# x, y positions only} \\

\noindent \textit{\# compute distances and apply Gaussian noise} \\
r = path[:, None, :] - beacons[None, :, :] \\\\
$\eta$ = $\sqrt{\mathrm{variances}} * \mathcal{N}(0, 1)$ \\\\
$\tilde{d} = \|r\| + \eta$ 

\begin{itemize}
  \item output: $out := \tilde{d} \in \mathbb{R}^{p \times N}$ (noisy range measurements from robot to each beacon)
  \item exception: None
\end{itemize}

\noindent fm\_robot(trajectory, fm\_map): \\

\noindent \textit{\# extract robot and FM pose in map frame} \\
$(x_r, y_r, \theta_r) =$ trajectory \\
$(x_f, y_f, \_) =$ fm\_map \\

\noindent \textit{\# transform FM pose to robot frame} \\
dx := $x_f - x_r$ \\
 dy := $y_f - y_r$ \\
$x_{rel} := \cos(\theta_r) \cdot dx + \sin(\theta_r) \cdot dy$ \\
$y_{rel} := -\sin(\theta_r) \cdot dx + \cos(\theta_r) \cdot dy$ \\
$\phi := \arctan2(y_{rel}, x_{rel})$ 

\begin{itemize}
  \item output: $out := T_{rf} \in \mathbb{R}^3$ (relative FM pose in robot frame)
  \item exception: None
\end{itemize}


\noindent visible\_fms(robot\_pose, fm\_map, max\_range, fov\_angle): \\

\noindent \textit{\# initialize list of visible FMs} \\
visible := [~] \\

\noindent \textit{\# check distance and FOV for each FM} \\
for each $(i, fm)$ in fm\_map: \\
\hspace*{1em} $(x_{rel}, y_{rel}, \phi) := fm\_robot(robot\_pose, fm)$ \\\\
\hspace*{1em} if $\|[x_{rel}, y_{rel}]\| \leq$ max\_range and $0 \leq \phi \leq$ fov\_angle: \\\\
\hspace*{2em} visible.append((i, $(x_{rel}, y_{rel}, \phi)$)) 

\begin{itemize}
  \item output: $out :=$ visible of $(\text{index}, T_{rf} \in \mathbb{R}^3)$ (visible FMs in robot frame)
  \item exception: None
\end{itemize}

% \wss{A module without environment variables or state variables is unlikely to
%   have a state transition.  In this case a state transition can only occur if
%   the module is changing the state of another module.}

% \wss{Modules rarely have both a transition and an output.  In most cases you
%   will have one or the other.}

\newpage

\section{MIS of Localization Module} \label{M_Localize}

\subsection{Module}

localization

\subsection{Uses}
\begin{itemize}
  \item GTSAM Module (Section \ref{M_GTSAM})
  \item Simulation Module (Section \ref{M_Simulate})
  
\end{itemize}
\subsection{Syntax}

\subsubsection{Exported Constants}
None

\subsubsection{Exported Access Programs}

\begin{center}
\begin{tabular}{p{2cm} p{7cm} p{4cm} p{3cm}}
\hline
\textbf{Name} & \textbf{In} & \textbf{Out} & \textbf{Exceptions} \\
\hline
localize & $\mathbf{a}:\mathbb{R}^{N \times 2}, \mathbf{T}_{mf}:\mathbb{R}^3, \mathbf{T}_{rf}: \mathbb{R}^3,\mathbf{\tilde{d}}:\mathbb{R}^N$ & $\mathbb{R}^3$ & - \\
\hline
\end{tabular}
\end{center}

\subsection{Semantics}

\subsubsection{State Variables}
None

\subsubsection{Environment Variables}
None

\subsubsection{Assumptions}
None

\subsubsection{Access Routine Semantics}

\noindent localize($\mathbf{a}$, $\mathbf{T}_{mf}$ $\mathbf{\tilde{d}}, \mathrm{initial\_guess}$):\\

\noindent \textit{\# initialize factor graph and robot symbol}\\
graph = NonlinearFactorGraph()\\
robot\_id = symbol("x", 1)\\

\noindent \textit{\# add prior on robot pose}\\
graph.add(PriorFactorPose2(robot\_id, $x_0$, prior\_noise))\\

\noindent \textit{\# add range factors to visible beacons}\\
for each (j, position, $d_j$) in visible\_beacons:\\
    graph.add(RangeFactor2D(robot\_id, symbol("a", j+1), $d_j$, range\_noise))\\

\noindent \textit{\# fix one beacon position for stability}\\
graph.add(PriorFactorPoint2(symbol("a", 1), position\_1, beacon\_noise))\\

\noindent \textit{\# add between factors from visible fiducial markers}\\
for each fiducial i:\\
    graph.add(BetweenFactor(robot\_id, symbol("f", i+1), $T_{rf}$, pose\_noise))\\
    graph.add(PriorFactorPose2(symbol("f", i+1), $T_{mf}$, pose\_noise))\\

\noindent \textit{\# initialize estimates}\\
insert robot, beacon, and fiducial guesses into Values()\\

\noindent \textit{\# optimize with LM}\\
result = LevenbergMarquardtOptimizer(graph, initial\_estimates)\\
$\hat{x}$ := result.atPose2(robot\_id)



\begin{itemize}
% \item transition: \wss{if appropriate} 
\item output: $out:= \mathbf{\hat{x}} \in \mathbb{R}^3$
\item exception: ValueError if estimation fails or result is not computable
\end{itemize}

% \subsubsection{Local Functions}
% None
\newpage

\section{MIS of Accuracy Evaluation Module} \label{M_Accurate}

\subsection{Module}

accuracy

\subsection{Uses}
\begin{itemize}
  \item Localization Module (Section \ref{M_Localize})
\end{itemize}


\subsection{Syntax}

\subsubsection{Exported Constants}
None

\subsubsection{Exported Access Programs}

\begin{center}
\begin{tabular}{p{4cm} p{5cm} p{4cm} p{2cm}}
\hline
\textbf{Name} & \textbf{In} & \textbf{Out} & \textbf{Exceptions} \\
\hline
compute\_fim & $\hat{\mathbf{x}}: \mathbb{R}^3, \mathbf{a}: \mathbb{R}^{N \times 2}, \boldsymbol{\sigma^2}: \mathbb{R}^N$  & $\mathbb{R}^{2 \times 2}$ & - \\
compute\_crlb & $\boldsymbol{\mathcal{I}}(\hat{\mathbf{x}}): \mathbb{R}^{2 \times 2}$ & $\mathbb{R}^{2 \times 2}$ & - \\
\hline
\end{tabular}
\end{center}

\subsection{Semantics}

\subsubsection{State Variables}
None

\subsubsection{Environment Variables}
None

\subsubsection{Assumptions}
\begin{itemize}
  \item Noise variances are positive
\end{itemize}

\subsubsection{Access Routine Semantics}

\noindent compute\_fim($\hat{\mathbf{x}}, \mathbf{a}, \boldsymbol{\sigma^2}$):
\begin{itemize}
% \item transition: \wss{if appropriate} 
\item output: $out:=\boldsymbol{\mathcal{I}}(\hat{\mathbf{x}})$ where $\boldsymbol{\mathcal{I}}(\hat{\mathbf{x}})$ is a \( 2 \times 2 \) Fisher Information Matrix (FIM) of the estimated pose, computed as:
\[
\mathcal{I}(\hat{\mathbf{x}}) = \sum_{j=1}^{N} \frac{1}{\sigma_j^2} \frac{(\hat{\mathbf{x}} - \mathbf{a}_j)(\hat{\mathbf{x}} - \mathbf{a}_j)^T}{\|\hat{\mathbf{x}} - \mathbf{a}_j\|^2}
\]
where $\mathbf{\hat{x}}$ only contains $x$ and $y$ (making it $\mathbb{R}^2$ so it can subtract)
\item exception: None
\end{itemize}

\noindent compute\_crlb($\boldsymbol{\mathcal{I}}(\hat{\mathbf{x}})$):
\begin{itemize}
% \item transition: \wss{if appropriate} 
\item output: $out:=$ A \( 2 \times 2 \) CRLB matrix, computed as:
\[
\boldsymbol{\mathcal{C}} = \boldsymbol{\mathcal{I}}^{-1}
\]
\item exception: None
\end{itemize}

% \subsubsection{Local Functions}
% None

\newpage

\section{MIS of Output Module} \label{M_Output} 
\subsection{Module}

output

\subsection{Uses}
\begin{itemize}
  \item Localization Module (Section \ref{M_Localize})
  \item Python multiprocessing Library (Queue)
\end{itemize}

\subsection{Syntax}

\subsubsection{Exported Constants}
None

\subsubsection{Exported Access Programs}

\begin{center}
\begin{tabular}{p{4cm} p{4cm} p{4cm} p{2cm}}
\hline
\textbf{Name} & \textbf{In} & \textbf{Out} & \textbf{Exceptions} \\
\hline
update\_table & - & - & - \\
run\_gui & queue: Queue & - & - \\
\hline
\end{tabular}
\end{center}

\subsection{Semantics}

\subsubsection{State Variables}
\begin{itemize}
  \item root: Tkinter root window
  \item tree: Treeview widget for table display
  \item data\_queue: Shared queue received from Control Module
\end{itemize}
\subsubsection{Environment Variables}
\begin{itemize}
  \item data\_queue: A queue storing tuples of estimated pose data (time, x, y, theta).
  \item display\_env: The OS-level display environment variable required to render the Tkinter GUI (e.g., `\$DISPLAY` for Unix-based systems).
\end{itemize}

\subsubsection{Assumptions}
\begin{itemize}
  \item The function `run\_gui()` is executed in a separate process to prevent a stalled execution.
\end{itemize}
\subsubsection{Access Routine Semantics}

\noindent update\_table():
\begin{itemize}
    \item transition: Retrieves the latest pose estimates from the queue and updates the Graphical User Interface (GUI) table.
\end{itemize}

\noindent run\_gui(queue):
\begin{itemize}
    \item transition: Initializes and runs the Tkinter GUI while continuously checking for pose updates.
\end{itemize}

\noindent \textit{\# Start GUI loop}\\\\
root = tk.Tk()  \textit{\# Root GUI window}\\\\
tree = ttk.Treeview(\dots)  \textit{\# Table structure for displaying (pose\_num,$ x, y, \theta$)}\\\\

\noindent \textit{\# Periodically check the shared queue}\\\\
root.after(100, update\_table)  \textit{\# 100 ms update interval}\\\\
root.mainloop()


% \subsubsection{Local Functions}
% None

\newpage

\section{MIS of Plotting Module} \label{M_Plotting}

\subsection{Module}

plot 

\subsection{Uses}
\begin{itemize}
  \item Localization Module (Section \ref{M_Localize})
\end{itemize}

\subsection{Syntax}

\subsubsection{Exported Constants}
None

\subsubsection{Exported Access Programs}

\begin{center}
\begin{tabular}{p{4cm} p{6cm} p{4cm} p{2cm}}
\hline
\textbf{Name} & \textbf{In} & \textbf{Out} & \textbf{Exceptions} \\
\hline
plot\_localization\_live & $\mathbf{a}: R^{N \times 2}, \mathbf{T}_{mf}: R^3$, map: String (filepath), show: $\mathbb{B}$  & - & - \\
\hline
update\_trajectory &  $\hat{\mathbf{x}}: R^3$ & - & - \\
\hline
\end{tabular}
\end{center}

\subsection{Semantics}

\subsubsection{State Variables}
\begin{itemize}
  \item trajectory: Global sequence updated by `update\_trajectory()' from Control Module (section \ref{M_Control}).
\end{itemize}

\subsubsection{Environment Variables}
\begin{itemize}
  \item plot\_env: The Matplotlib interactive rendering backend required to run real-time plotting.
\end{itemize}

\subsubsection{Assumptions}
\begin{itemize}
  \item \lq plot\_localization\_live()\rq~is run in an interactive Matplotlib session.
  \item \lq update\_trajectory()\rq~is only called when valid estimated poses exist.
  \item `plot\_env' supports `plt.ion()' and `plt.pause()' for animation updates.
  \item localize()(Section \ref{M_Localize}) either returns a valid pose or raises an exception.
\end{itemize}

\subsubsection{Access Routine Semantics}

\noindent plot\_localization\_live($\mathbf{a}, \mathbf{T}_{mf}$, map, \textbf{path}, map\_size, show=True):

\noindent \textit{\# render map background and plot fixed elements} \\
img = plt.imread(map) \\
plt.imshow(img, extent=[0, map\_size[0], 0, map\_size[1]]) \\
plt.plot(a[:, 0], a[:, 1], 'o') \ \textit{\# beacons} \\
plt.plot($T_{mf}$[:, 0], $T_{mf}$[:, 1], 'x') \ \textit{\# fiducial markers} \\
plt.plot(path[:, 0], path[:, 1], '--') \ \textit{\# ground truth trajectory} \\

\noindent \textit{\# initialize plot elements} \\
robot\_trajectory, = plt.plot([], [], 'r-') \\
triangle\_patch = Polygon([], closed=True, color='red') \\
plt.gca().add\_patch(triangle\_patch) \\

\noindent \textit{\# begin animation loop} \\
ani = FuncAnimation(fig, update, interval=200) \\
if show: plt.show()

\begin{itemize}
  \item transition: Initializes and continuously updates a real-time localization plot.
  \item output: None (renders a live plot in Matplotlib)
  \item exception: RuntimeError if the Matplotlib backend does not support animation
\end{itemize}

\noindent update\_trajectory($\hat{\mathbf{x}}$):
\begin{itemize}
\item transition: Called repeatedly by Matplotlib's `FuncAnimation' to retrieve and render the latest estimated pose from the global `trajectory'
\end{itemize}



\subsubsection{Local Functions}
\noindent update(frame):\\
\noindent \textit{\# update robot path and triangle orientation} \\
$x, y, \theta$ := trajectory[frame] \\
robot\_trajectory.set\_data(... up to frame ...) \\
triangle\_patch.set\_xy(...) \textit{\# rotated triangle}

\begin{itemize}
    \item transition: Retrieves the latest estimated pose from the trajectory and updates the visualization.
\end{itemize}

\newpage


\bibliographystyle {plainnat}
\bibliography {../../../refs/References}

% \newpage

% \section{Appendix} \label{Appendix}

% \renewcommand{\arraystretch}{1.2}

% \begin{longtable}{l p{12cm}}
% \caption{Possible Exceptions} \\
% \toprule
% \textbf{Message ID} & \textbf{Error Message} \\
% \midrule

% \end{longtable}

% \wss{Extra information if required}

\end{document}